\documentclass[a4paper,12pt]{report}

\usepackage{color}
\usepackage{mathtools}
\usepackage[brazilian]{babel}
\usepackage[utf8]{inputenc}
\usepackage[T1]{fontenc}
\usepackage{tikz}
\usetikzlibrary{arrows.meta}
\usetikzlibrary{automata,positioning}
\usepackage{pgfplots}
\usepackage{filecontents}
\usepackage{cancel}
\usetikzlibrary{arrows.meta}
\usetikzlibrary{arrows.meta}
\usepackage{bm}
\usepackage{mathrsfs}
\usepackage{blkarray}
\usepackage{gensymb}
\usepackage{graphicx}
\graphicspath{/Users/gustavo/Desktop/Estatística\ Matemática}
\usepackage{amssymb}
\usepackage{tkz-euclide}
\usepackage[margin=0.1in]{geometry}
\usepackage{enumitem} 
\usepackage{accents}


\author{}
\geometry{textwidth=6in, textheight=9in, marginparsep=7pt, marginparwidth=.6in, top=30mm, bottom=25mm}

\title{Exercícios Inferência Matemática\\
Barry R. James - Probabilidade: um curso em nível intermediário\\
Capítulo 1 e Capítulo 2
}
\date{}
\begin{document}
	\maketitle
	\tableofcontents	
	\newpage
	
	
	\chapter{Páginas 88- 32}
	\section*{Questão 17}
	 \begin{enumerate}[label=\alph*)]
	 	\item 
	 	$$P(0<X\le 1 , 0 <Y\le 1) = F(1,1) - F(1,0)-F(0,1)+F(0,0) $$
	 	$$1-e^{-2} - (1-e^{-1})  - (1-e^{-1}) + 1 = -2e^{-1} -e^{-1}\not>0 $$
	 	Logo, a função não é função de distribuição de um vetor aleatório
	 	
	 	\item Para demonstrar temos que avaliar que a função é não decrescente, absolutamente contínua, tem valor 0 quando x e y tendem a $-\infty$ e valor 1 quando x e y tendem a $\infty$ e provar que  o  retângulo gerado pela distribuição em $\mathbb R^2$ tem área maior ou igual a 0.
	 \end{enumerate}
	 		\section*{Questão 18}

 	
	 	\begin{enumerate}[label=\alph*)]
	 		\item  $P(X=x,Y=y)=\begin{cases}
	 			0, & x=y\\
	 			\frac{1}{6}, & x\ne y
	 		\end{cases}, \ \ \ \  x,y=1,2,3$
	 			 	\item  $P(X<Y) = \frac{1}{2}$
	 	\end{enumerate}
	 	
	\section*{Questão 19}
	
		\section*{Questão 20}
		\begin{enumerate}[label=\alph*)]
			\item $f(x,y)=\begin{cases}
			\frac{1}{\pi}, & -1\le x,y\le 1\\
			0, & c.c
			\end{cases}$
			\item $P(X>Y)$ Corresponde á metade inferior da área do circulo a partir da reta x=y, portanto, a probabilidade equivale a  $\frac{1}{2}$. O mesmo é válido para $P(X<Y)$, que correndo a metade superior.
			
			O caso onde P(X=Y) é uma reta e portanto não determina uma região a ser calculada, resultado em probabilidade 0
		\end{enumerate}
	
			\section*{Questão 21}
			O problema trata do produto de duas variáveis uniformes contínuas, então:
					\begin{enumerate}[label=\alph*)]
						\item 
			$$f_(X,Y)(x,y)=\begin{cases}
			1, & x,y \in [0,1]\\
			0, & c.c
			\end{cases} $$
			\item $$P\bigg(|\frac{Y}{X}-1| \le \frac{1}{2}\bigg)  = 
			P\bigg(\frac{1}{2} X \le Y \le \frac{3}{2} X  \bigg)$$
			$$= \int\limits_{0}^{\frac{2}{3}}\int\limits_{\frac{x}{2}}^{\frac{3x}{2}}1 dydx +   
			\int\limits_{\frac{2}{3}}^{1}\int\limits_{\frac{x}{2}}^{1}1 dydx
			$$
			$$ 
=	\int\limits_{0}^{\frac{2}{3}}x dx +   
	\int\limits_{\frac{2}{3}}^{1}(1-\frac{x}{2}) dx = \frac{5}{12}
			$$
			\item $$P\bigg(Y\ge X| Y \ge \frac{1}{2}\bigg)  = P\bigg(
			\frac{P(Y\ge X, Y\ge \frac{1}{2})}{P(Y\ge \frac{1}{2})}
			\bigg)
			=
			\frac{\int\limits_{\frac{1}{2}}^1\int\limits_0^y 1 dxdy}{\frac{1}{2}} = 
				\frac{\int\limits_{\frac{1}{2}}^1 y dy}{\frac{1}{2}} = \frac{3}{4}
			$$
			\end{enumerate}
	 		\section*{Questão 22}

	\section*{Questão 23}
	
	Por definição, o conjunto de 3 variáveis aleatórias é dito independente se:
	
	\begin{itemize}
		 \item $P(X_i=x_i,X_j=x_j) = P(X_i=x_i)P(X_j=x_j)$, $\forall i\ne j  \ \ i,j=1,2,3$	
		 		 \item $P(X_i=x_i,X_j=x_j,X_k=x_k) = P(X_i=x_i)P(X_j=x_j)P(X_k=x_k)$, $\forall i\ne j \ne k\ \ i,j=1,2,3$	
	
	\end{itemize} 
Sendo assim, independência 2 a 2 não garante que as variáveis aleatórias são independentes.
\newpage 
	\section*{Questão 24}
	
	$$F_{X,Y}(x,y) = \begin{cases}
	(1-e^{-x})(1-e^{-y}), & x,y>0\\
	0, & cc
	\end{cases} $$
	
	$$
	f_{X,Y}(x,y)=\begin{cases}
	e^{-x}	e^{-y}, & x,y > 0\\
	0, & c.c
	\end{cases}
	$$
	
	$$
	f_X(x) = \int\limits_0^\infty 	e^{-x}	e^{-y} dy = 	e^{-x}\therefore X \sim Exp(1)
	$$
		$$
	f_Y(y) = \int\limits_0^\infty 	e^{-x}	e^{-y} dx = 	e^{-y}\therefore Y \sim Exp(1)
	$$
	
	X e Y são independentes pois o produto das marginais é igual a conjunta.
		\section*{Questão 25}	 	
  $$P(X=x,Y=y)=\begin{cases}
			0, & x=y\\
			\frac{1}{6}, & x\ne y
			\end{cases}, \ \ \ \  x,y=1,2,3$$

		
		
		$$P(X=x)= \begin{cases}
		\frac{1}{3}, & x=1,2,3\\
		0, & c.c
		\end{cases} $$
		
			
		$$P(Y=y)= \begin{cases}
		\frac{1}{3}, & y=1,2,3\\
		0, & c.c
		\end{cases} $$
		
		É fácil ver que o produto as marginais não vai ser igual a distribuição conjunta.
		\newpage 
		\section*{Questão 26}
		Seja $\bm X=(X_1,\ldots,X_n )$ um vetor aleatório com f.d.a $F_{\bm X}(x_1,\ldots,x_n)$\\
		Então:
		$$\lim\limits_{x_k\rightarrow \infty}F_{\bm X}(x_1,\ldots,x_{k-1},x_k,x_{k+1},\ldots,x_n)$$
		$$ =
		\lim\limits_{n\rightarrow \infty} P\bigg(
		X_1^{-1}\bigg((-\infty,x_1]\bigg),\ldots, \underbrace{X_k^{-1}\bigg((-\infty,\cancelto{\infty}{x_k}]\bigg)}_{\Omega},\ldots, X_n^{-1}\bigg((-\infty,x_n]\bigg)
		\bigg)$$
		$$= \underbrace{F_{X_1\ldots,X_{k-1},X_{k+1},\ldots X_n}(x_1,\ldots,x_{k-1},x_{k+1},\ldots,x_n)}_{\text{Distribuição marginal de ordem n-1}} $$ 
		Em particular,
		$$\bm X=(X,Y) $$
		$$ \underbrace{F_{X,Y}(x,y) = P(X\le  x,Y\le y)}_{\text{f.d.a conjunta de X e Y}} $$
		$$\Rightarrow \lim\limits_{x\rightarrow \infty} F_{X,Y}(x,y)=F_Y(y) $$
		$$\Rightarrow \lim\limits_{y\rightarrow \infty} F_{X,Y}(x,y)=F_X(x) $$
		São distribuições marginais de Y e X respectivamente.
		
		Tendo em vista que $$f_X(x) = \frac{\partial F_X(x)}{\partial x }  $$ 
		
		A proposição está provada, pois:
		
			$$ \lim\limits_{x\rightarrow \infty} F_{X,Y}(x,y)=F_Y(y)\Rightarrow \frac{\partial F_Y(y)}{\partial x } = \int\limits_{-\infty}^{+\infty} f_{X,Y}(x,y)dx $$
						$$ \lim\limits_{y\rightarrow \infty} F_{X,Y}(x,y)=F_X(x)\Rightarrow \frac{\partial F_X(x)}{\partial x }  = \int\limits_{-\infty}^{+\infty} f_{X,Y}(x,y)dy $$
						
						\newpage 
						
								\section*{Questão 27}
								
								$$Xt^2+Yt+Z = 0 $$
								$$\Delta  = Y^2-4XZ\ge 0 $$
								$$P(\Delta \ge 0 ) = P(Y^2 - 4XZ\ge 0 ) $$
								$$P(Y^2 \ge  4XZ) = P(XZ\le \frac{Y^2}{4}) $$
								Utilizando a lei da probabilidade total
								$$P(XZ\le \frac{Y^2}{4})=P(X\le K,Z\le \frac{Y^2}{4K}) $$
								Como X,Y,Z são independentes, temos:
								$$P(X\le K,Z\le \frac{Y^2}{4K}) = P(X\le K)P(Z\le\frac{Y^2}{4K} ) $$
								$$= K\int\limits_0^{\frac{y^2}{4K} } dz = \int\limits_0^y K \frac{y^2}{4K} dy= \frac{1}{12}$$
							\section*{Questão 28}
							\newpage 
														\section*{Questão 29}
														$$P(X=c,X=c)=P(X=c)P(X=c) = P(X=c)= \bigg[
														P(X=c)
														\bigg]^2 $$
														Logo, $P(X=c) =\begin{cases}
														1, & x=c\\
														0, & x\ne c
														\end{cases}$
														
														Logo, X é independente de si mesmo se, e somente se, for uma função indicadora.
														
	\section*{Questão 30}
		\begin{enumerate}[label=\alph*)]
				\item $$P(T<t)=P(T\le t ,T=T_1) +P(T\le t ,T=T_2)  $$
				$$P(T=t_1)P(T\le t|T=t_1) + P(T=t_2)P(T\le t|T=t_2)$$
				$$= \frac{1}{2}P(T_1\le t) +\frac{1}{2}P(T_2\le t)   $$
				$$ = \frac{1}{2}\bigg(1-e^{t\lambda_1} +1-e^{t\lambda_2}\bigg)
				 = 1 - \frac{1}{2}\bigg(e^{t\lambda_1} +e^{t\lambda_2}\bigg)
				$$
				
				\item 
				
				$$P(T=t_1|T\ge 100) = \frac{P(T=t_1,T\ge 100)}{P(T\ge 100)}
				= \frac{P(T=t_1)P(T\ge 100|T=t_1)}{P(T\ge 100)} 
				 $$
				 $$
				 = \frac{e^{-\lambda_1 \cdot 100}}{e^{-\lambda_1 \cdot 100}+e^{-\lambda_2 \cdot 100}}
				 $$
				 \item $$T\sim exp(\lambda)$$
					\end{enumerate}
				
					\section*{Questão 31}
					
					$$P(T_1>2T_2) = \int\limits_0^\infty P(T_1>2t|T_2=t)dt $$
					$$= \int\limits_{0}^{\infty}\int\limits_{2t_2}^{\infty}f(t_1,t_2)dt_1 dt_2 = 
					\int\limits_{0}^{\infty}\int\limits_{2t_2}^{\infty}f(t_1)f(t_2)dt_1 dt_2
					 $$
					 $$
					 \int\limits_{0}^{\infty}\lambda e^{-\lambda t_2} \int\limits_{2t_2}^{\infty}\lambda e^{-\lambda t_1} dt_1 dt_2
					 = \int\lim\limits_{0}^\infty \lambda e^{t_2\lambda}e^{-2t_2}dt_2 = \frac{1}{3}
					 $$
														\newpage 
														
	\section*{Questão 32}
		\begin{enumerate}[label=\alph*)]
				\item 
						
Nota-se que estamos tratando uma região (A) formada pela conjunto de duas distribuições  de amplitude 2, sendo assim, a densidade é dada por:
																		
				$$	f_{X,Y}(x,y) = \begin{cases}
					\frac{1}{2}, &  (x,y)\in A\\
				0, & (x,y)\not\in A
	\end{cases}
				$$
				\item 
					$$f(x) =\begin{cases}
				\int\limits_{-1-x}^{x+1} \frac{1}{2} dy, & =1\le x < 0\\
					\int\limits_{x-1}^{1-x} \frac{1}{2} dy, & 0\le x \le 1
			\end{cases}$$
																		
			$$
				f(x) = \begin{cases}
			1+x, &-1\le x < 0\\
			1-x, & 0\le x < 1
			\end{cases}
			$$
																		Da mesma forma,
																		
																	$$
																	f(y) = \begin{cases}
																	1+y, &-1\le y < 0\\
																	1-y, & 0\le y < 1
																	
																	\end{cases}
																	$$	
																	\item É fácil ver que o produto das marginais não resulta na distribuição conjunta.
		\end{enumerate}
																						
				\section*{Questão 33}
									\begin{enumerate}[label=\alph*)]
										\item															
				$$ 
			P(X=x)=\begin{cases}
					\frac{1}{5}, & x=1 \\
					\frac{3}{5}, & x=2 \\
					\frac{1}{5}, & x=3\\
																																												
\end{cases}
						$$
						
										$$ 
						P(Y=y)=\begin{cases}
						\frac{1}{5}, & y=1 \\
						\frac{3}{5}, & y=2 \\
						\frac{1}{5}, & y=3\\
						\end{cases}
						$$
						\item
						Não não independentes, pois:
						
						$$P(X=2,Y=2)=\frac{1}{5} \ne P(X=2)P(Y=2)= \frac{9}{25} $$
						
						
							\end{enumerate}
						
						\newpage 
						
							\section*{Questão 34}
							
							$$Z=X-Y $$
							
							$X=I_{[\theta - 0.5,\theta+0.5]}$ e $-Y= I_{[-\theta - 0.5,-\theta+0.5]}$
							
							 $$P(X-Y\le z) = \begin{cases}
							 \int\limits_{\theta - 0.5}^{\theta+0.5+z} \int\limits_{x-z}^{\theta+0.5}1dydx, & -1 < z < 0\\
							 1 - \int\limits_{z}^{\theta+0.5}\int\limits_{\theta-0.5}^{x-z}1dydx, & 0\le z< 1
							 \end{cases} 
							 $$
							 $$
							 =
							 \begin{cases}
							 [\theta + 0.5+z-(\theta-0.5)][\theta+0.5-(\theta-0.5-z)] & -1 < z < 0\\
							 	 1 -[\theta+0.5-z-(\theta-0.5)][\theta+0.5-(\theta-0.5+z)], & 0\le z< 1
							 \end{cases}
							 $$
							 	 $$
							 =
							 \begin{cases}
							 \frac{(1+z)^2}{2} & -1 < z < 0\\
						1-\frac{(1-z)^2}{2}, & 0\le z< 1
							 \end{cases}
							 $$
Logo
$$ 
f_Z(z)=\begin{cases}
						1+z,& -1 < z < 0\\
1-z, & 0\le z< 1
\end{cases}
$$
					\section*{Questão 35}
					 \begin{enumerate}[label=\alph*)]
					 	\item 
					 	
					 	$$P(X_i^2\le y) = 
					 	P(X_i \le \sqrt y)
					 =
					 \int\limits_0^{\sqrt y} \frac{t}{\theta^2} e^{-\frac{t^2}{2\theta^2}} dt 
					 	 $$
					 	 $$=  
					 	 \int\limits_0^{ y} \frac{1}{2\theta^2} e^{-\frac{\mu}{2\theta^2}} du = F_Y(y):Y\sim Exp\bigg(\frac{1}{2\theta^2}\bigg)
					 	 $$
					 	 $$f_{\bm Y }(\bm y) =\bigg(\frac{1}{2\theta^2}\bigg)^n e^{-\frac{\sum\limits_{i=1}^{n}y_i}{2\theta^2}}, \forall y_i>0,i=1,\ldots n  $$
					 	 \item 
					 	 $$P(U\le u) = P(Min(\bm X)\le u)) = 1- P(\bm X > u) $$
					 	 
					 	 $$= 1-\prod\limits_{i=1}^{n}P(X_i > u) = 1-[P(X_i > u)]^n $$
					 	 
					 	 $$= 1-\bigg[\int\limits_u^\infty \frac{x}{\theta^2} exp\bigg(-\frac{x^2}{2\theta^2}\bigg)dx \bigg]^n
					 	 = 1-\bigg[\int\limits_{u^2}^\infty \frac{1}{2\theta^2} exp\bigg(-\frac{t}{2\theta^2}\bigg)dt \bigg]^n
					 	  $$
					 	  $$
					 	  = 1 - \bigg(exp\bigg\{-\frac{\mu^2}{2\theta^2}\bigg\}\bigg)^n = 1 - exp\bigg\{-\frac{n\mu^2}{2\theta^2}\bigg\}
					 	  \Rightarrow f_U(u) = \begin{cases}
					 	  \frac{nu}{1\theta^2} e^{-\frac{nu}{2\theta^2}}, &u \ge 0\\
					 	  0, &c.c
					 	  \end{cases}
					 	  $$
					 	  \item $$P(Z\le z) = P(X\le zX_2) = \int\limits_0^\infty \int\limits_\frac{t}{z}^\infty f_{X,Y}(s,t)dsdt $$
					 	  $$ \int\limits_0^\infty \int\limits_\frac{t}{z}^\infty \frac{st}{\theta^4} exp\bigg\{
					 	  -\frac{s^2+t^2}{2\theta^2}
					 	  \bigg\} dsdt  
					 	  = \frac{z^2}{1+z^2}, z\ge 0
					 	  $$
					 \end{enumerate}

\end{document}