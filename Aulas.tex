\documentclass[a4paper,12pt]{article}

\usepackage{color}
\usepackage{mathtools}
\usepackage[brazilian]{babel}
\usepackage[utf8]{inputenc}
\usepackage[T1]{fontenc}
\usepackage{tikz}
\usetikzlibrary{arrows.meta}
\usetikzlibrary{automata,positioning}
\usepackage{pgfplots}
\usepackage{filecontents}

\usetikzlibrary{arrows.meta}
\usetikzlibrary{arrows.meta}
\usepackage{bm}
\usepackage{mathrsfs}
\usepackage{blkarray}
\usepackage{gensymb}
\usepackage{graphicx}
\usepackage{amssymb}
\usepackage{tkz-euclide}
\usepackage{geometry}
\usepackage{enumitem} 
\usepackage{cancel}

\author{}
\geometry{textwidth=6in, textheight=9in, marginparsep=7pt, marginparwidth=.6in, top=30mm, bottom=25mm}

\title{Estatística Matemática}
\date{}
\begin{document}
	\maketitle
	\tableofcontents	
	
	\newpage
	
	\section*{Biliografia Importante}
	
	kai Lai Chung: Introduction to Probability\\
	Barry R. James: Probabilidade : um Curso Em Nível Intermediário 
	
	\newpage
	\section{ Aula 13/03}
	\subsection{Espaço de Probabilidade $(\Omega,\mathscr{A},\mathscr{P})$ }
\underline{Experimento Aleatório($\varepsilon$) }\\
\begin{itemize}
	\item Pode ser repetido infinitas vezes se for necessário;
	\item Permite descrever todos os possíveis resultados;
	\item Quando repetido um numero grande de vezes os resultados apresentam certa regularidade.($\Rightarrow$  Probabilidade resultante)
\end{itemize}
\underline{Exemplos}:

\begin{enumerate}[label=(\arabic*)]
	\item Lancar um dado e observar sua face superior.
	\item Lancar duas moedas honestas e observar suas faces superiores.
	\item Retirar  6 bolas de uma urna com 60 bolas numeradas (Sem reposição e sem ordem).
	\item Observar o valor de um retorno financeiro.
	\item Contar o numero de carros que chegam a UnB no primeiro dia de aula.
\end{enumerate}

\underline{Definição} ( Espaço Amostral)\\

Define-se como espaço amostral o conjunto de todos os possíveis resultados do experimento.
\\

\underline{Notação}:\\
\\
$\Omega=\{\omega:\omega \ \text{é resultado de } \Omega\}$\\
$\Omega$: Espaço amostral associado a $\varepsilon$.\\
$\varepsilon$: Experimento Aleatório.\\
\\
\underline{Exemplos}\\
\\
$\varepsilon_1 \rightarrow \Omega_1=\{1,2,3,4,5,6\}, \ \ \ \ \ |\Omega_1|=6 $\\
\\
$\varepsilon_2 \rightarrow \Omega_2=\{(c,c),(c,\hat{c}),(\hat{c},c),(\hat{c},\hat{c})\},  \ \ \ \ \ |\Omega_2|=4 $\\
\\
$\varepsilon_2 \rightarrow \Omega_2'=\{(i,j): i,j \in \{c,\hat{c}\}\}, \ \ \ \ \ |\Omega_2|=4  $\\
\\
$\varepsilon_3 \rightarrow \Omega_3=\{\{b_1,\dots,b_6\}: b_1,\dots,b_6 \in \{1,..,60\},b_i\ne b_j \forall i\ne j, \ i,j =1,\ldots,6\},\\
|\Omega_3| = \binom{60}{6} 
 $\\
 \\
$\varepsilon_3' \rightarrow \Omega_3'=\{(b_1,\dots,b_6): b_1,\dots,b_6 \in \{1,..,60\},b_i\ne b_j \forall i\ne j, \ i,j =1,\ldots,6\},\\
|\Omega_{3'}| = A_{60,6} = \frac{60!}{54!}
$\\
 \\
$\varepsilon_3'' \rightarrow \Omega_3''=\{\{b_1,\dots,b_6\}: b_1,\dots,b_6 \in \{1,..,60\}\}\\
|\Omega_{3''}| = 60^6
$\\
 \\
$\varepsilon_3''' \rightarrow \Omega_3'''=\{(b_1,\dots,b_6): b_1,\dots,b_6 \in \{1,..,60\}\}\\
|\Omega_{3'''}| = \frac{60^6}{6!}
$\\
 \\
$\varepsilon_4 \rightarrow \Omega_4=\{0,1,2,\ldots,\} = \mathbb{Z^+}\cup \{0\}\\
|\Omega_{4}| = \infty
$\\
 \\
$\varepsilon_5 \rightarrow \Omega_5=(-\infty,\infty)=\{r:r\in \mathbb{R} \}, r=P_t-P_{t-1}, P=\text{preço da ação}5
$\\

	\section{ Aula 18/03 }
	
	\underline{Definição}
	
	Seja $\Omega$ um espaço amostral de $\varepsilon$. A colecao de eventos de $\Omega$ é uma $\sigma-$algebra se satisfaz:
	
	\begin{enumerate}[label=\roman*)]
		\item $\mathscr{A} \ne \emptyset$
		
		\item $A \in \mathscr{a}\Rightarrow A^c =\Omega \textbackslash A \in \mathscr{A}  $ 
		
		\item $A_1,A_2,\ldots,\in \mathscr{A} \Rightarrow \bigcup\limits_{i=1}^{\infty}A_i\in \mathscr{A}$
 	\end{enumerate}
 
 \underline{Exemplo}:\\
 \\
 A= Sair o mesmo resultado  = $\{(c,c,),(\hat c,\hat c) \}$
 \newpage
 	\section{ Aula 20/03}
 	
 	$$\varepsilon \rightarrow \Omega = \{\omega: \ \  \omega \ \text{ponto amostral (resultado de} \ \varepsilon)\} $$
 	
 	$\mathscr{A} $=\{ Eventos de $\Omega$ \} Tal que:
 	\begin{enumerate}[label=\roman*)]
\item $\mathscr{ A} \ne \emptyset$
\item $A \in \mathscr{A} \Rightarrow A^c \in \mathscr{A}$
\item $A_1,A_2, \ldots \in \mathscr{A}\Rightarrow \bigcup\limits_{i=1}^\infty A_i \in \mathscr{A}$
 	\end{enumerate}

 $\sigma$-algebra de eventos de $\Omega$\\
 \\
\subsection{Medida de Probabilidade}

Seja $\mathscr{A}$ uma $\sigma$-algebra de eventos de $\Omega$. A função $P:\mathscr{A}\rightarrow [0,1]$ é uma \underline{medida de probabilidade}  se for não negativa e $\sigma$-aditiva, Isto é, satisfaz:

\begin{enumerate}[label=\roman*)]
	\item $P(A)\ge 0, \forall A\in \mathscr{A} $
	\item $P(\Omega)=1$
	\item $A_1,A_2,\ldots \in \mathscr{A}(A_i\cap A_j=\emptyset, \forall i\ne j )$
	$\Rightarrow P(\bigcup\limits_{i=1}^{\infty}A_i) = \sum\limits_{i=1}^\infty  P(A_i)$: $\sigma$-aditividade
\end{enumerate}

\underline{Notação}:
$\varepsilon\longrightarrow (\Omega,\mathscr{A},P) $ Espaço de probabilidade de $\varepsilon$\\
\\
\underline{Observação}: $(\Omega,\mathscr{A},P)$
\begin{itemize}
	\item $\Omega$ enumerável $\Rightarrow (\Omega,\mathscr{A},P)$ é discreto.
	\item $\Omega$ não enumerável $\Rightarrow (\Omega,\mathscr{A},P)$ é contínuo.
	
\end{itemize}

\underline{Exemplo:} $\Omega_5,\Omega_6$
\newpage
\subsection{Propriedade da medida de Probabilidade}

\begin{enumerate}[label=\textbf{\arabic*)}]
	\item $P(\emptyset)=0 $\\
 $\Omega\cup\emptyset=\Omega $\\
 $\Rightarrow P(\Omega) +P(\emptyset)=P(\Omega)$\\
 $\Rightarrow 1 +P(\emptyset)=1 \Rightarrow P(\emptyset)=0$
 \item $P(A^c)=1-P(A)$\\
 $A\cup A^c=\Omega $\\
 $\Rightarrow P(A)+P(A^c)=P(\Omega)\Rightarrow P(A)=1-P(A^c) $
 \item $P(A\cup B)=P(A)+P(B)-P(A\cap B)$\\
 $$A\cup B=A\cup[B\textbackslash A\cap B] \Rightarrow P(A\cup B)=P(A)+P(B\textbackslash A\cap B) \ \ \ \ (1)$$
 Por outro lado,\\
 $B=[B\textbackslash A\cap B]\cup(A\cap B)$\\
 $\Rightarrow P(B) = P(B\textbackslash A\cap B) +P(A\cap B)$\\
 $P(B\textbackslash A\cap B)=P(B) - P(A\cap B) \ \ \ \ \ (2) $\\
 Substituindo 2 em 1,  o resultado está provado.
 \item $P(\bigcup\limits_{i=1}^{n}A_i)=\sum\limits_{i=1}^n P(A_i)- \sum\limits_{i\ne j \le n}P(A_i\cap A_j)+\ldots+(-1)^{n+1}P(\bigcap\limits_{i=1}^nA_i) $\\
 \underline{Prova por indução}:\\
 
 $n=2: \ P(A_1\cup A_2)=P(A_1)+P(A_2)-P(A_1\cap A_2) $ é válida.\\
 Hipótese de indução: $P(\bigcap\limits_{i=1}^k A_i)=\sum\limits_{i=1}^{k}P(A_i)- \sum\limits_{i\ne j \le k}P(A_i\cap A_j)+\ldots+(-1)^{k+1}P(\bigcap\limits_{i=1}^kA_i)  $\\
 Supor que a hipótese é válida\\
 \\
 Provar que: $P(\bigcap\limits_{i=1}^{k+1}A_i)=\sum\limits_{i=1}^{k+1}P(A_i)- \sum\limits_{i\ne j \le k+1}P(A_i\cap A_j)+\ldots+(-1)^{k+2}P(\bigcap\limits_{i=1}^{k+1}A_i)  $\\
 \underline{Observação}:
 $$P(\bigcup\limits_{i=1}^3 A_i)=P(A_1\cup A_2 \cup A_3)=P(A_1\cup A_2)+P(A_3)-P([A_1\cup A_2]\cap A_3) $$
 $$=P(A_1)+P(A_2)-P(A_1\cap A_2)+P(A_3)- P([A_1\cap A_3]\cup [A_2\cap A_3]) $$
 $$=P(A_1)+P(A_2)+P(A_3 )- P(A_1\cap A_2)- \bigg\{ P(A_1\cap A_3)+P(A_2\cap A_3) - P((A_1\cap A_3)\cap (A_2 \cap A_3))  \bigg\} $$
 $$=\sum\limits_{i=1}^3 P(A_i)-P(A_1\cap A_2) - P(A_1\cap A_3) - P(A_2\cap A_3) + P(A_1\cap A_2\cap A_3) $$
  $$=\sum\limits_{i=1}^3 P(A_i)-\sum\limits_{i\ne j \le 3}P(A_i\cap A_j) +P(A_1\cap A_2\cap A_3) $$
\end{enumerate}

\newpage
\subsection{Cálculo de Probabilidade}

\begin{enumerate}[label = \alph*)]
	\item $(\Omega,\mathscr{A},P)$ discreto com resultados equiprováveis e finito.\\
	$\Omega = \{\omega_1,\omega_2,\ldots,\omega_n\}$\\
	$\forall \omega_i \in \Omega, \ \ P(\omega_i)=\frac{1}{|\Omega|}=\frac{1}{n}$\\
	Então, $A=\{\omega_1,\ldots, \omega_k  \} \subset \Omega, k\le n$\\
	$A=\bigcup\limits_{i=1}^k\{\omega_i\}\Rightarrow P(A) = \sum\limits_{i=1}^{k} P(\omega_i) \Rightarrow P(A) = \sum\limits_{i=1}^{k} \frac{1}{n}=\frac{k}{n}$ ou $P(A)=\frac{|A|}{|\Omega|} $\\
	\\
	\underline{Exemplo}:\\
	
	\begin{enumerate}[label=\textbf{\arabic*)}]
		\item 
	
	De um lote de 100 computadores, no qual sabe-se que há 10 defeituosos, são escolhidos aleatoriamente  5 computadores sem reposição.
	Qual é a probabilidade de sairem 3 computadores defeituosos na amostra?\\
	
	\underline{Solução}:
	$$\Omega = \bigg\{\omega= \{c_1,\ldots,c_5 \}: c_i \in \{1,\ldots,100\}, c_i\ne c_j , \forall i\ne j,\ \ \ i,j=1,2,3,4,5  \bigg\} $$
	$$ |\Omega| =\binom{100}{5}   $$
	$$A = \bigg\{\omega= \{c_1,\ldots,c_5 \} \in \Omega: 3 c_i's \in \{10\ defeituosos \}, 2 c_i's \in \{90 \ nao \ defeituosos\}$$
	$$\Rightarrow |A| = \binom{10}{3}\cdot\binom{90}{2} $$
	$$\Rightarrow P(A) = \frac{\binom{10}{3}\binom{90}{2}}{\binom{100}{5}} $$
	
	\item $\varepsilon_3 \rightarrow \omega_3, \mathscr{A}=P(\Omega_3)$    $ |\Omega| = \binom{60}{6}$
	\begin{itemize}
		\item 
	
	A =Obter os números sorteados na megasena.
	$$ |A| = \binom{6}{6}\binom{54}{0} $$
	$$\Rightarrow P(A) = \frac{1}{\binom{60}{6}} $$
	
	\item  B = Obter uma quina com os números sorteados da megasena
		$$ |B| = \binom{6}{5}\binom{54}{1} $$
			$$\Rightarrow P(B) = \frac{\binom{6}{5}\binom{54}{1}}{\binom{60}{6}} $$
\end{itemize}
\end{enumerate}
\newpage
\item $(\Omega,\mathscr{A},P)$ discreto e infinito.
$\Omega = \{ \omega_1,\omega_2,\ldots\}$\\
Tem-se que, $\forall \omega_i \in \Omega $,\\
$p_i=P(\omega_i)$ é tal que $\sum\limits_{i=1}^{\infty} p_i=1$\\
Daí,\\
$A=\{w_1,\ldots,w_k\} \in \Omega, k \in \mathbb{Z}^+$\\
então $P(A) = P(\bigcup\limits_{i=1}^{k}\omega_i) = \sum\limits_{i=1}^{k} p_i $\\
\\
\underline{Exemplo}:\\
$\varepsilon_4 \rightarrow \Omega_4=\{0,1,2,\ldots\}$\\
Suponha que os carros chegam conforme
$$P(\omega_i)=p_i=\frac{e^{-\lambda}\lambda^i}{i!}, \lambda \in \mathbb{R}^+ $$
Note que:
$$ \sum\limits_{i=1}^{\infty}p_i = \sum\limits_{i=1}^{\infty} \frac{e^{-\lambda}\lambda^i}{i!}=1$$

 A = \{Chegam no mínimo 2 carros\}
 $$\Rightarrow P(A)=1-P(A^c) = 1-P(\{0,1\}) = 1 - \bigg[ p_0 + p_1 \bigg] $$

\newpage

\section{Aula 25/03}
\subsection{Espaço de Probabilidade contínuo}
	$\varepsilon\longrightarrow(\Omega,\mathscr{A},P)$ Espaço de Probabilidade contínuo.\\
\\
$P:\mathscr{A}\longrightarrow[0,1]$ é medida de probabilidade se existir uma função $f:\mathbb{R}\rightarrow\mathbb{R}$ t.q.

\begin{enumerate}[label=\alph*)]
	\item Se: $ \Omega \subseteq \mathbb{R},f(x)\ge 0$ e $\int\limits_{\Omega} f(x)dx<\infty $ (f-Integrável ou f é  mensurável com respeito á $\mathscr{A}$), $f:\mathbb{R}\longrightarrow\mathbb{R}$ e  $\forall A\in\mathscr{A},$
	 $P(A)=\frac{\int\limits_{A}f(x)dx}{\int\limits_{\Omega}f(x)dx} = \int\limits_{A} \bigg(\frac{f(x)}{\int\limits_{\Omega} f(x)dx}\bigg)dx$,
	 $f_1(x)=\frac{f(x)}{\int\limits_{\Omega}f(x)dx} f.densidade $
	 
	 \underline{Exemplo}: Medida de probabilidade uniforme ( Eventos equiprováveis)
	 
	 $\Omega =[a,b], a<b\in \mathbb{R}, \mathscr{A}=\mathscr{B}\bigg([a,b]\bigg). \forall A\subset [a,b]=\Omega$\\
	 \\
	 $P(A) = \frac{Com(A)}{comp(\Omega)}= \frac{\int\limits_A 1dx}{\int\limits_a^b dx}, f(x)\equiv 1 $\\
$=\int\limits_{A} \frac{1}{b-a} dx, f_1(x) = \begin{cases}
\frac{1}{b-a}, x\in[a,b]\\
0 \ \ \ , c.c
\end{cases}$

\item Se $\Omega \subset \mathbb{R}^2$, f é t.q. $f(x,y)\ge0$ e $\iint\limits_{\Omega } f(x,y) dx dy <\infty$.\\ Então, $\forall A\in \mathscr{A}$,

$$P(A)=\iint\limits_{A}\frac{f(x,y)}{\iint\limits_{\Omega} f(x,y)} dxdy $$

\underline{Exemplo}: 
$$\varepsilon_6 \rightarrow \Omega=S^2=\{(x,y)\in \mathbb{R}^2: x^2+y^2\le 1\} \, \mathscr{A}=\mathscr{B}(S^2),\forall A\in \mathscr{A},$$
$$P(A)=\frac{area(A)}{area(\Omega)}= \frac{area(A)}{\pi} $$
$$=\iint\limits_{A} \frac{1}{\pi}dxdy, \ \ f(x,y)= \begin{cases}
\frac{1}{\pi}, (x,y)\in S^2\\
0\  \ \ , c.cc
\end{cases}=\frac{1}{4}
$$
\end{enumerate}
\newpage
Pois,

$$P(A) = \frac{1}{\pi} \int\limits_{0}^1 \sqrt{1-x^2} dx =\frac{1}{\pi}\int\limits_{0}^{\frac{pi}{2}} cos\theta\cdot\cos\theta d\theta $$
$$=\frac{1}{\pi} \int\limits_{0}^{\frac{pi}{2}} cos^2(\theta) d\theta$$
$$\frac{1}{2\pi} \bigg[\theta - \frac{1}{2}(sen 2\theta)\bigg]\bigg\vert^{\frac{\pi}{2}}_{0}$$
$$=\frac{1}{2\pi}\bigg[\bigg(\frac{\pi}{2}-\frac{1}{2}(0)\bigg)-(0-0)\bigg] = \frac{1}{4} $$

\underline{OBS}: (Medida uniforme em $\mathbb{R}^n$)
$$\Omega \subset \mathbb{R}^n, \mathscr{A=\mathscr{B}(\Omega)}$$
$$f(x)=\begin{cases}
\frac{1}{Volume(\Omega)}, x\in \mathbb{R}^n\\
0\ \ \ \ \ \ \ \  \ \  \ ,c.c
\end{cases} $$

A = retângulo de $\mathbb{R}^n=\prod\limits_{i=1}^n[a_i,b_i], a_i<b_i$
\newpage
\subsection{Outras propriedades da medida de probabilidade}

$P_1:$   $A\subset B \Rightarrow P(A)\le P(B) $\\
\\
$P_2:$  Continuidade da probabilidade.\\
\\
\begin{enumerate}[label=\alph*)]
	\item Seja $\{A_n\}_{n\ge 1} $ uma sequência crescente de eventos, t.q. \\$\displaystyle{\lim_{n \to \infty}} A_n=A (A_n \uparrow A)$, Então:

$$P(A)=P(\displaystyle{\lim_{n\to \infty}} A_n)= \displaystyle{\lim_{n \to \infty}} P(A_n)$$
\underline{Prova:} Por hipótese, temos que:
$$A_1\subset A_2\subset \ldots \subset A_n \subset \ldots \subset A$$
Então, temos que:

$$A_n=A_1\sqcup(A_2\textbackslash A_1)\sqcup(A_3\textbackslash A_2)\sqcup\ldots\sqcup(A_n\textbackslash A_{n-1})  $$
$$A=A_1\sqcup(A_2\textbackslash A_1)\sqcup\ldots\sqcup(A_n\textbackslash A_{n-1})\sqcup \ldots $$
Por outro lado, como

$$\displaystyle{\lim_{n \to \infty}} A_n=A, $$
Temos que:

$$P(A) = P(\displaystyle{\lim_{n \to \infty}}A_n) $$
$$P(A) = P\bigg(\bigsqcup\limits_{i=1}^\infty (A_i\textbackslash A_{i-1})\bigg) \overbrace{=}^{\sigma-aditivdade}
\sum\limits_{i=1}^{\infty}P(A_i\textbackslash A_{i-1}) = \displaystyle{\lim_{n \to \infty}}  \sum\limits_{i=1}^{\infty} P(A_i\textbackslash A_{i-1})
$$

$$ 
=\displaystyle{\lim_{n \to \infty}}  P\bigg(\bigsqcup\limits_{i=1}^\infty (A_i\textbackslash A_{i-1})\bigg)  = \displaystyle{\lim_{n \to \infty}} P(A_n)
$$



\item Seja $\{A_n\}_{n\ge 1} $ uma sequência crescente de eventos  t.q. \\$\displaystyle{\lim_{n \to \infty}} A_n=A (A_n \downarrow A)$, Então:
$$P(A)=P(\displaystyle{\lim_{n \to \infty}} A_n)= \displaystyle{\lim_{n \to \infty}} P(A_n)$$
\end{enumerate}



\underline{Exemplo}:

$\varepsilon$: lançar um dado indefinidamente.\\
\\
Qual a probabilidade de não sair 1 ou 6 em nenhum lançamento.\\
\\
\underline{Solução}:\\
A = não sair 1 ou 6 em nenhuma jogada.\\
Defina:\\
$A_n$ = não sair 1 ou 6 até o n-ésima jogada.\\
Temos que:
$$P(A_n)= \bigg(\frac{4}{6}\bigg)^n$$
e

$$A_1\supset A_2 \ldots \supset A_n \supset \ldots \supset \bigcap\limits_{i=1}^{\infty}A_i= A$$
Então, pela continuidade da probabilidade,
$$P(A)=\displaystyle{\lim_{n \to \infty}}  P(A_n) = \displaystyle{\lim_{n \to \infty}} \bigg(\frac{4}{6}\bigg)^n=0 $$

$A^c:$ não sair 1 ou 6 em alguma jogada:
$$P(A^c)=1-P(A)=1 $$

\end{enumerate}

\newpage

\subsection{Limite Superior e inferior de Eventos}

\underline{Definição}: Seja $\{A_n\}_{n\ge 1}$ uma sequência de eventos no espaço de probabilidade $(\Omega,\mathscr{A},P)$. Definem-se os
seguintes eventos:
\begin{enumerate}[label=\alph*)]
	\item $\limsup\limits_{n\rightarrow \infty} A_n = \bigcap\limits_{n=1}^\infty\bigg(\bigcup\limits_{i=n}^\infty A_i\bigg)$ (= $A_n$ infinitas vezes)
		\item $\liminf\limits_{n\rightarrow \infty} A_n = \bigcup\limits_{n=1}^\infty\bigg(\bigcap\limits_{i=n}^\infty A_i\bigg)$ (= $A_n$ algumas vezes)
\end{enumerate} 

\underline{Observação 1}:\\
\\

Se $\limsup\limits_{n\rightarrow \infty} A_n=\liminf\limits_{n\rightarrow \infty} A_n$ então $\{A_n\} $ possui limite e 
$$\displaystyle{\lim_{n \to \infty}}  A_n=\limsup\limits_{n\rightarrow \infty} A_n=\liminf\limits_{n\rightarrow \infty} A_n $$
\underline{Observação 2}:\\
\\
$$\omega  \in\limsup\limits_{n\rightarrow \infty} A_n = \bigcap\limits_{n=1}^\infty\bigg(\bigcup\limits_{i=n}^\infty A_i\bigg) $$
$$\Rightarrow  \omega \in \bigcup\limits_{i=n}^\infty A_i, \forall n\ge 1$$
$$\Rightarrow\omega \in A_{i_{n_1} },\forall n_1\ge n $$
$$\Rightarrow  \omega \in \bigcup\limits_{i=n_1+1}^\infty A_i,$$
$$\Rightarrow\omega \in A_{i_{n_2 }},\forall n_2> n1 $$
$$\Rightarrow\omega \in A_{i_{n_k }},\forall n_k>\ldots> n2>n_1 $$
Ou seja
$$\omega \in A_{i_{n_1} },A_{i_{n_2} },\ldots $$
$\omega $ está em infinitos eventos dessa sequência
$$ \Leftrightarrow \displaystyle{\limsup_{n \to \infty}}  A_n = \{A_n\  \text{Infinitas vezes}\}$$
\newpage

\underline{Observação 3}:\\
\\
\begin{itemize}
	\item 

$$\limsup\limits_{n\rightarrow \infty} A_n = \bigcap\limits_{n=1}^\infty\bigg(\bigcup\limits_{i=n}^\infty A_i\bigg), B_n = \bigcup\limits_{i=n}^\infty A_i $$

$$B_1\supset B_2 \ldots \supset B_n \supset \ldots \supset \bigcap\limits_{i=1}^{\infty}B_n= \bigcap\limits_{n=1}^\infty\bigg(\bigcup\limits_{i=n}^\infty A_i\bigg)=\limsup\limits_{n\rightarrow \infty} A_n$$


\item 
$$\liminf\limits_{n\rightarrow \infty} A_n = \bigcup\limits_{n=1}^\infty\bigg(\bigcap\limits_{i=n}^\infty A_i\bigg)$$

$$B_1\subset B_2 \ldots \subset B_n \supset \ldots \supset \bigcup\limits_{i=1}^{\infty}B_n= \bigcup\limits_{n=1}^\infty\bigg(\bigcap\limits_{i=n}^\infty A_i\bigg)=\liminf\limits_{n\rightarrow \infty} A_n$$

\end{itemize}

\newpage
\subsection{Aula 27/03}

Questões da prova

\begin{itemize}
	\item Formular um espaço amostral para um experimento
	\item demonstrar alguma propriedade de probabilidade
	\item liminf ou limsup
	\item Probabilidade condicional
\end{itemize}
	\subsection{Probabilidade condicional}
	
	Seja $(\Omega,\mathscr{A},P)$ o espaço de probabilidade.\\
	\\
	\underline{Definição}: Sejam $A,B\in \mathscr{A}$ tais que $P(B)>0$, define-se a probabilidade condicional de A dado B como:
	
	$$P(A|B)=\frac{P(A\cap B)}{P(B)} $$
	\underline{Observação}:
	Note que $\forall B\in \mathscr{A}$, define-se:\\
	\\
	$P_B:\underset{A\longrightarrow P_B(A)}{\mathscr{A}\longrightarrow[0,1] }$ é uma medida de probabilidade,\\
	\\
	em que $P_B(A) = P(A|B)= \frac{P(A\cap B)}{P(B)}$ \\
	\\
	Pois $P_B$ satisfaz (i)-(iii)
	
	\begin{enumerate}[label=(\roman*)]
		\item $$P_B(A)\ge 0, \forall A\in \mathscr{A} $$
		De fato,
		$$P_B(A)=\frac{P(A\cap B)}{P(B)}\ge 0 $$
		Pois
		$$P(A\cap B)\ge 0 $$
		
		\item $P_B(\Omega)=1 $
		
		\item $A_1,A_2,\ldots \in \mathscr{A} $ Eventos disjuntos
		$$\Rightarrow P_B\bigg( \bigcup\limits_{i=1}^\infty A_i \bigg) = \sum\limits_{i=1}^{\infty} P_B(A_i) $$
	\end{enumerate}

\newpage
\underline{Exemplo}:
$\varepsilon $: Lançar dois dados e observar as faces superiores\\
\\
A = ambos resultados são diferentes\\
\\
B= Um dos resultados é 6. \\
\\
Calcule:\\
\begin{enumerate}[label=\roman*)]
	\item $$P(A|B); P(B|A) $$
	$$\Omega=\bigg\{(i,j),\forall i,j\in \{1,\ldots,6\}\bigg\} $$
	$$A=\{(i,j)\in \Omega: i\ne j \} $$
	$$B=\{(i,j)\in \Omega: \{i,j\}\in {6}\cap \{i+j<12\} \} $$
	$$P(A)=\frac{30}{36} $$
	$$P(B)=\frac{11}{36} $$
	$$P(A\cap B)=\frac{10}{36} $$
\end{enumerate}

\newpage

\underline{Proposição}:\\
\\
Seja $(\Omega,\mathscr{A},P)$ o espaço de probabilidade e $A_1,\ldots,A_n\in \mathscr{A} $. Então:

\begin{enumerate}[label=\arabic*)]
	\item $P(A_1\cap,\ldots,\cap A_{n-1}\cap A_n) $\\
	$$=P(A_1)\cdot P(A_2|A_1)\ldots P(A_n|A_1\cap\ldots\cap A_{n-1}) $$
	
	\underline{Prova}:\\
	\\
	Por indução:
	
	$$h=2:\ \ \ \ P(A_1\cap A_2)=P(A_1)\cdot P(A_2|A_1) $$
	$$h=k: \ \ \ \ P(A_1\cap \ldots \cap A_k) = P(A_1)\ldots P(A_k|A_1\cap \ldots \cap A_{k-1}) $$
		$$P(A_1\cap \ldots \cap A_k\cap A_{k+1}) = P(A_1)\ldots P(A_k|A_1\cap \ldots \cap A_{k}) $$
		Basta considerar 
		$$P(\underbrace{A_1\cap \ldots \cap A_k}_{A}\cap A_{k+1})  $$
		$$\overset{n=2}{=} P(A_1\cap \ldots \cap A_k)\cdot P(A_{k+1}|A_1\cap \ldots \cap A_k) $$
			$$\overset{(H.I.)}{=} P(A_1)\cdots P(A_k|A_1\cap \ldots \cap A_k)\cdot P(A_{k+1}|A_1\cap \ldots \cap A_k) $$
		
	
	\item Para $B\in \mathscr{A}$ e $\Omega=\bigcup\limits_{i=1}^n A_i$, com $A_i$ disjuntos.\\
	\\
	Isto é, $ \{A_i \}$ é uma partição  de $\Omega$, tem-se:
	
	
	$$P(B) = \sum\limits_{i=1}^{\infty} P(B\cap A_i) $$
	
	\item $$P(A_i|B)=\frac{P(B\cap A_i)}{P(B)} = \frac{P(B\cap A_i)}{\sum\limits_{i=1}^{\infty}P(B\cap A_i)} $$
	


\end{enumerate}


\newpage
\underline{Exemplo}:\\
\\
Considere uma urna com 10 bolas em que 3 são brancas, 5 azuis e 2 vermelhas.\\
Escolhem-se 4 bolas aleatoriamente, sem reposição. Qual a probabilidade de saírem 2 bolas azuis na amostra?\\
\\
$A_i$ = \{Obter duas bolas azuis \}\\

$$P(A)=\frac{\binom{2}{5}\binom{2}{5}}{\binom{4}{10}} $$

\newpage

\underline{01/04}

\begin{enumerate}[label=\alph*)]
	\item Os eventos $A,B\in \mathscr{A} $, são independentes se:
	$$P(A\cap B)=P(A)P(B)=P(A)P(B\textbackslash A) $$
	
	\item Os eventos $A_1,\ldots,A_n$ são mutuamente independentes,se:
	
	$$P(A_1\cap \ldots \cap A_k)=P(A_1)\ldots P(A_k), \forall k\le n $$

\underline{Exemplo, n=3}:
$$P(A_1\cap A_2)=P(A_1)P(A_2) $$
$$P(A_2\cap A_3)=P(A_2)P(A_3) $$
$$P(A_1\cap A_3)=P(A_1)P(A_3) $$
$$P(A_1\cap A_2\cap A_3)= P(A_1)P(A_2)P(A_3) $$

\underline{Exemplo}:\\
\\
$\varepsilon$: Escolher um ponto no retângulo $ [0,1]\times [0,1]$
$$\Omega = [0,1]\times [0,1] \{(x,y)\in \mathbb{R}: a\le x,y\le 1 \}$$

$A=$ a primeira componente é igual a segunda componente..
$B=$ a distância da origem ao ponto escolhido é menor de 0.5.\\
\\
A e B são independentes?

$$P(A) = \frac{1}{2} $$
$$P(B)=\frac{1}{4}\pi r^2=\frac{pi}{16}$$
$$\Rightarrow P(A\cap B)= \frac{\pi}{32}=P(A)P(B) =\frac{1}{2} \frac{\pi}{16}$$
$\therefore $ A e B são independentes.\\
\\


\underline{Exercícios}:\\ 
\\
\begin{enumerate}
	\item Dê um contra exemplo para a afirmação seguinte:\\
  "Eventos independentes 2-2 são mutuamente independentes  "
  \item $\varepsilon$ lançar dados e observar suas faces superiores:\\
  Defina:\\
  \\
  $A_1$ = sair par no primeiro dado\\
  $A_2$ = sair ímpar no primeiro dado\\
  $A_3$ = soma dos resultados é par\\
  São $A_1,A_2,A_3$ mutuamente independentes?
\end{enumerate}

\end{enumerate}


\subsection{Variável aleatória}
Considere $(\Omega,\mathscr{A},P)$ um espaço de probabilidade.\\
Informalmente, diz-se que uma função\\
\\
$X: \underset{\omega \longrightarrow X(\omega)=x}{\Omega \longrightarrow \mathbb{R}}$ é variável aleatória (v.a.)\\
\\
$\{X=x\}=\{\omega: X(\omega)=x \} $ for um característico numérico de pontos de $\Omega$.\\
\\
\underline{Exemplo}:\\
\\
$\varepsilon:$ Lançar 2 dados e observar suas faces superiores.\\
Se o interesse for de calcular a probabilidade associada á soma dos resultados, então:

$$|\Omega|=36,X: \underset{\omega \longrightarrow X(\omega)=x}{\Omega \longrightarrow \{2,3,\ldots,12\}} $$
$X$= soma dos resultados
$$\{X=2\}=\{\omega \in (i,j)\in \Omega: X(\omega)=2 \} $$
$$=\{\omega=(i,j)\in \Omega:i+j=2 \} $$
$$=\{(1,1)\} $$
$$\{X=3\}=\{(1,2),(2,1)\} $$

\subsubsection{Imagem Inversa}

$$f: \mathbb{R}_D\longrightarrow \mathbb{R}_I, \ \forall B \subset \mathbb{R}_I, $$
$$\bigg[f(A)=B \bigg]  \ \ \ ou \ \ \ f(B)=A$$
\begin{center}é o conjunto imagem inversa de B \end{center}

\underline{Propriedades:}\\
\\
\begin{enumerate}[label=\alph*)]
	\item $\bigg[f^{-1}(B) \bigg]^c = f^{-1}(B^c)$
	
	\item $f^{-1}(\bigcup\limits_{i=1}^{n}B_i) = \bigcup\limits_{i=1}^{n}f^{-1}(B_i)$
\end{enumerate}

\underline{Definição: Variável aleatória}\\
\\
Seja $(\Omega,\mathscr{A},P)$ um espaço de probabilidade. A função $X:\Omega \longrightarrow \mathbb{R} $ é uma variável aleatória, se:\\
\\
$\forall  B \in \mathscr{B}(\mathbb{R}) , X^{-1}(B)\in \mathscr{A}$, sendo $\mathscr{B}(\mathbb{R})$ a $\sigma$-algebra dos Borelianos de $\mathbb{R}$\\
\\
\underline{Note que}: $\mathscr{B}(\mathbb{R}) $ é $\sigma$-algebra pois:
\begin{enumerate}[label=\roman*)]
\item $\mathscr{B}(\mathbb{R}) \supset \mathbb{R},\mathscr{B}(\mathbb{R})\ne \emptyset$
\item  $B \in \mathscr{B}(\mathbb{R})$, sempre que $X^{-1}(B)\in \mathscr{A}$
$$\Rightarrow \bigg[ X^{-1}(B)\bigg]^c \in \mathscr{A} $$
$$\Rightarrow \bigg[ X^{-1}(B^c)\bigg] \in \mathscr{A} $$
$$\Rightarrow B^c \in \mathscr{B}(\mathbb{R}) $$
\item Exercício*



\end{enumerate}

\underline{Obs}: 
$$X:\underbrace{\Omega}_{A=X^{-1}(B)\in \mathscr{A}} \longrightarrow \underbrace{\mathbb{R}}_{B \in \mathscr{B}( \mathbb{R})}  $$

\underline{Definição 2}: Variável aleatória
Seja $(\Omega,\mathscr{A},P)$ um espaço de probabilidade, a função $X:\Omega \longrightarrow \mathbb{R} $ é variável aleatória se:

$$\forall x \in \mathbb{R}, X^{-1}\bigg( (-\infty.x] \bigg) \in \mathscr{A} $$
$$\Leftrightarrow \forall x \in \mathbb{R}, B=(-\infty,x]$$
$$\forall x \in \mathbb{R},   X^{-1}\bigg( (-\infty.x] \bigg) = \{ \omega \in \Omega: X(\omega)\in (-\infty,x] \}$$
$$= \{\omega \in \Omega: X(w)\le x \} $$
$$\{ X\le x \}\in \mathscr{A}$$

\newpage
\underline{Exemplo}:\\
\\
Seja $\Omega = \{1,2,3,4\}$ e $\mathscr{A} = \bigg\{\emptyset,\Omega, \underbrace{\{1,2\}}_{A} , \underbrace{\{3,4\}}_{A^c} \bigg\}$\\
\\
Verifique se $I_A(\omega) = \begin{cases}
1,\omega \in A\\
0, \omega \in A^c
\end{cases} $ 
\\
e $I_B(\omega) = \begin{cases}
1,\omega \in B\\
0, \omega \in B^c
\end{cases} $, $B=\{1,4\}$\\
\\
São variáveis aleatórias.\\
\\

\underline{Solução}:

$$I_A^{-1}\bigg ( (-\infty,x] \bigg) = \begin{cases}
\emptyset, x<0\\
A^c=X^{-1}(\{0\}), x\le 0 <1\\
X^{-1}(\{0\})\cup X^{-1}(\{1\}), x\ge 1
\end{cases} \in \mathscr{A} $$

$\therefore  \ I_A$ é uma variável aleatória

$$I_B^{-1}\bigg ( (-\infty,x] \bigg) = \begin{cases}
\emptyset, x<0\\
B^c=X^{-1}(\{0\})=\{2,3\}, x\le 0 <1\\
B\cup B^c, x\ge 1
\end{cases} \notin \mathscr{A} $$

\underline{Obs}:
\begin{itemize}
	\item 

Se a $\sigma$-algebra $\mathscr{A}$ conter $B$ e $B^c$, $I_B$ seria variável aleatória 
\item 
Em geral, se $\Omega = \{\omega_1,\ldots, \omega_n\}$ e $\mathscr{A}=P(\Omega)$, toda  função  $X:\Omega \longrightarrow \mathbb{R} $
é variável aleatória.
\item 
$\Omega = \mathbb{R}$ e X é uma função contínua então  X é variável aleatória
\item 

$\Omega = \mathbb{R}$ ,  $X:\Omega \longrightarrow \mathbb{R} $ é variável aleatória $\Leftrightarrow$ X é Borel Mensurável

\item 
Seja $X:\Omega$  $X:\Omega \longrightarrow \mathbb{R} $  variável aleatória e g Borel mensurável, então
$X\circ g$ é uma variável aleatória\\
$X\circ g :\Omega$  $X:\Omega \longrightarrow \mathbb{R}$

\end{itemize}
\newpage
\underline{Definição:}Medida de Probabilidade:
Seja $(\Omega,\mathscr{A}=\sigma(x),P_X)$ o espaço de probabilidade. A função \\
$P_X: \mathscr{A}\longrightarrow[0,1] $ é medida de probabilidade.\\

\underline{Prova}:
\begin{enumerate}[label=\roman*)]
	\item 

$P_X(B)=P\bigg(X^{-1}(B) \bigg)=P\bigg([X\in B]\bigg)\ge 0, \forall B \in \mathscr{B}(\mathbb{R}) $

\item $P_X(\mathbb{R})=P\bigg(X^{-1}(\mathbb{R}\bigg) = P(\Omega)=1 $

\item $B_1,B_2,\ldots \in \mathscr{B}(\mathbb{R})$ disjuntos,

$$P_X\bigg(\bigcup\limits_{i=1}^{\infty}B_i\bigg)= 
P_X\bigg(X^{-1}\bigg(\bigcup\limits_{i=1}^{\infty}B_i\bigg)\bigg)
 $$
 
 $$
 P_X\bigg(\bigcup\limits_{i=1}^{\infty}X^{-1}\bigg(B_i\bigg)\bigg)
 $$
 
 $$
 =\sum\limits_{i=1}^{\infty}P(X^{-1}(B_i))
 $$
 
 $$=\sum\limits_{i=1}^{\infty}P_X(B_i) $$
 
\end{enumerate}

\underline{Obs}:

$$B=(-\infty,x] \in \mathscr{B}(\mathbb{R}), \forall x\in \mathbb{R} $$
$$\Rightarrow P_X(B)= P_X\bigg( (-\infty,x])\bigg) $$
$$= P\bigg(X^{-1}\bigg(-\infty,x]\bigg)\bigg) $$
$$=P\bigg( x\in (-\infty,x] \bigg) = P(X\le x) = F_X(x)	 $$

\newpage

\section{03/04}

\subsection{Variável Aleatória }
Revisão:\\
\\


$(\Omega,\mathscr{A},P)$ \\
Uma função 
$$X:\Omega \longrightarrow \underbrace{I}_{\text{conjunto de valores de X}}\subset \mathbb{R}\ \text{é variável aleatória}$$

\begin{itemize}
	\item  $$\forall B \in \ \mathscr{B}(\mathbb{R}) , \ A=X^{-1}(B) = [ X\in B]\in \mathscr{A}$$
	$$X^{-1}(B)=\{\omega \in \Omega: X(\omega)\in B \} $$
	$$\Rightarrow P\bigg(X^{-1}(B)\bigg) = P(X\in B) \ \ \underline{f.d.a}$$
	$$(\Omega,\underbrace{\mathscr{A}}_{A \in x^{-1}(B)},P) \longrightarrow(\mathbb{R},\underset{\in B}{\mathscr{B}(\mathbb{R})},p_x) $$
\item $B=(-\infty,x]$, X é variável aleatória se $\forall x \in \mathbb{R}$,
$$X^{-1}\bigg( (-\infty,x] \bigg) = [X\le x]\in \mathscr{A}, \Rightarrow P(X\le x)=P_X\bigg( (-\infty,x] \bigg) =F_X(x) $$
$$X^{-1}\bigg( (-\infty,x] \bigg) =\{\omega \in \Omega: X(\omega)\in (-\infty,x]\} $$
$$=\{\omega \in \Omega: X(\omega)\le x \}  $$
\end{itemize}

\newpage
\subsection{Tipos de Variáveis aleatórias}

\begin{enumerate}
	\item Uma  variável aleatória  $X$ é discreta se o conjunto dos seus valores for enumerável
	$$\underset{\omega: \longrightarrow X(\omega)=x_i}{X:\Omega \longrightarrow\{x_1,x_2,\ldots\}} $$
	Neste caso, $P_X: \mathscr{B}(\mathbb{R}) \rightarrow [0,1]$, satisfaz:
	\begin{enumerate}[label=\roman*)]
		\item $P_x(\{X_i\})= P\bigg(X^{-1}(\{x_i\}) \bigg)=P(X=x_i)\ge 0$;
		\item $\sum\limits_{i\ge 1}P_X(\{x_i\})=1$
	\end{enumerate}
Então $P_x$ é conhecida como função de probabilidade da variável aleatória X.\\
\\
Note que:\\
\\
$1=P(\Omega)=P\bigg( \bigcup\limits_{i\ge 1} X^{-1}(\{x_i\})\bigg)=\sum\limits_{i\ge 1}P_X(x_i)$\\
ou\\
$1=P\bigg( \bigcup\limits_{i\ge 1} X^{-1}(\{x_i\})\bigg)=\sum\limits_{i\ge 1}P_X(x_i) = P\bigg(\bigcup\limits_{i\ge 1} [X=x_i]\bigg)
=P\bigg( X\in \{x_1,x_2,\ldots\}\bigg)=\sum\limits_{i\ge 1} P_X(x_i)
$\\
	 X é variável aleatória com valores $\{x_1,x_2,\ldots\}$
	 $$\Leftrightarrow P\bigg( X\in \{x_1,x_2,\ldots\} \bigg) =1$$
	 \\
	 \\
	 \subsubsection{Fução de distribuição acumulada (f.d.a)}
	 Seja $X$ v.a. talve que $P\bigg( X\in \{x_1,x_2,\ldots\} \bigg) =1$\\
	 Então, $\forall b \in \mathscr{B}(\mathbb{R}),$
	 $$P_X(B)=P\bigg( X^{-1}(B) \bigg) = P(X\in B)= \sum\limits_{i:x_i\in B} P_X(x_i) $$
	 Em particular, $B=(-\infty,x], \forall x\in \mathbb{R}$,
	 $$F_x(x)= P_X\bigg( (-\infty,x]\bigg)=P\bigg( X^{-1}((-\infty,x])\bigg) =  \sum\limits_{i:x_i\in (-\infty,x]} P_X(x_i)   $$
\end{enumerate}

\newpage
\underline{Exemplo}:\\
\\
\underline{E.1} $X\sim B(n,p),n \in \mathbb{N}, 0<p<1 $\\
\\
$\varepsilon:\rightarrow\begin{cases}
	Sucesso = A, \ \ \ \ \ P(A)=p\\
	Fracasso = B, \ \ \ P(B)=1-p
\end{cases}$\\
Cada repetição de  $\varepsilon$ = Prova de Bernoulli\\
\\
\begin{enumerate}[label=\alph*)]
	\item 

$Y:\Omega \longrightarrow \{0,1\} $ é uma variável aleatória se:
$$P(X=0)=P_X(\{0\})=p $$
$$P(X=1)=P_X(\{1\})=1-p $$
Pois $P_X$ é função de probabilidade.\\
\\
Y é conhecida como variável aleatória $Bernoulli(p)$

\item  $$\underset{\omega \longrightarrow X(\omega)=k}{X:\Omega \longrightarrow \{0,1,\ldots,n\}}$$
X = número de sucessos em \underline{n provas} independentes de Bernoull, p = P(sucesso)\\
\\
Temos que:
$$P_X(k)=P\bigg( X^{-1}(\{k\})\bigg)=P\bigg(\{ \omega \in \Omega: X(\omega)=k \}\bigg)=P(X=k) $$

$$P\bigg( \bigg\{\omega =\{\underbrace{r_1,\ldots, r_k}_{\in \{A\}},\underbrace{r_{k+1},\ldots, r_n}_{\in A^c} \} \bigg\} \bigg) $$


$$\Omega = \bigg\{ \omega=\{ r_1,\ldots,r_n\}: r_i \in \{ A,A^c \},i=1,\ldots,n \bigg\}$$

$$\Rightarrow P_X(k)=\binom{n}{k}\cdot P\bigg( (\overbrace{A,\ldots,A}^{k},\overbrace{A^c,\ldots,A^c}^{n-k}) \bigg)  $$
$$= \binom{n}{k}P(A)\cdots P(A)\cdot P(A^c)\cdots P(A^c) $$
$$\Rightarrow P_X(k)= \binom{n}{k} p^k (1-p)^{n-k}, k=0,1,\ldots,n\ \ \ \ (*)$$
Como:

$$\sum\limits_{k=0}^{n}P_X(k) = \sum\limits_{k=0}^{n}\binom{n}{k} p^k (1-p)^{n-k} = \bigg[ (1-p)+p\bigg]^n=1 $$
Então $P_x$ dada em (*) é uma função de probabilidade de $X\sim B(n,p)$

\end{enumerate}
\newpage
\underline{Aplicação}:\\
\\

Suponha que para a megasena da Páscoa são vendidas 120 000 000 de cartelas.
\begin{enumerate}[label=\alph*)]
	\item 

 Qual é a probabilidade da megasena da páscoa ser ganha por 10 pessoas?\\
 
 \item 
 Qual a probabilidae da megasena ser ganha por no maximo 10 pessoas ?
 $$P(X\le 10) = \sum\limits_{k=0}^{10} \binom{n}{k} p^k(1-p)^{n-k}= P\bigg( X^{-1}\bigg((-\infty,10 ]\bigg)\bigg) $$
 
\end{enumerate}
\underline{Solução}: 
$$p=\frac{1}{50063860} $$
$$\Rightarrow P(X=10) = P(x^{-1}(10))=\binom{n}{10}p^{10}(1-p)^{n-10} $$


\newpage

Revisar as v.as
\begin{enumerate}
	\item  $X\sim Geo(p) $
	
	$$P_X(k|p)=P(X^{-1}({k}))=P\bigg(\{\omega \in \Omega:X(\omega)=k \}\bigg)=P(X=k)$$
$$P\bigg( \bigg\{\omega =\underbrace{\{f_1,f_2,\ldots}_{\in A^c}\underbrace{s_1}_{\in A} \} \bigg\} \bigg) $$
$$\Omega = \bigg\{ \omega= \{ f_1,f_2,\ldots,s_1\}: t_k \in \{A^c \},k=1,2,\ldots \ \ s_1 \in \{A\} \bigg\}$$
$$\Rightarrow P_X(k)=P\bigg( \overbrace{A^c,A^c,\ldots}^{k},\overbrace{A}^{1} \bigg)  $$
$$P(A)=p \Rightarrow P(A^c)=1-p$$
	$$\Rightarrow P_X(k|p)=(1-p)^{k}\cdot p, \ k=0,1,2,\ldots $$
	
	
	\item $X \sim Pascal (r,p) $
	
	$$P_X(k|p)=P(X^{-1}({k}))=P\bigg(\{\omega \in \Omega:X(\omega)=k \}\bigg)=P(X=k)$$
$$P\bigg( \bigg\{\omega =\underbrace{\{f_1,f_2,\ldots}_{\in A^c},\underbrace{s_1,s_2,\ldots}_{\in A}\} \bigg\} \bigg) $$
$$\Omega = \bigg\{ \omega= \{ f_1,f_2,\ldots,s_1,s_2,\ldots\}: f_k \in \{A^c \},k=1,2,\ldots \ \ s_i \in \{A\},i=1,2\ldots\bigg\}$$
$$\Rightarrow P_X(k)=P\bigg( \overbrace{A^c,A^c,\ldots}^{k},\overbrace{A,\ldots,A}^{i} \bigg)  $$
$$P(A)=p \Rightarrow P(A^c)=1-p$$
$$= \binom{i+k-1}{i}P(A)\cdots P(A)\cdot P(A^c)\cdots P(A^c) $$
$$\Rightarrow P_X(k)= \binom{i+k-1}{i} p^k (1-p)^{i}, k=0,1,\ldots \ \ i=1,2,\ldots$$
	\newpage
	\item $X\sim Hipergeo(N,r,p)$
		$$P_X(k|p)=P(X^{-1}({k}))=P\bigg(\{\omega \in \Omega:X(\omega)=k \}\bigg)=P(X=k)$$
		
		$$P\bigg(\omega=\bigg\{A_k\cap A_k^c:A_k= \{e_1,e_2,\ldots,e_j, \ j=0,1,\ldots,k \}, k=0,1,\ldots,i\bigg\},i=0,1,\ldots,N\bigg) $$
		$$K\le N $$
	$$\Omega = \bigg\{ \omega= (d_1,\ldots,d_n): d_i \in  :  \bigg\}$$
	


	$$= \binom{i+k-1}{i}P(A)\cdots P(A)\cdot P(A^c)\cdots P(A^c) $$
	$$\Rightarrow P_X(k)= \binom{i+k-1}{i} p^k (1-p)^{i}, k=0,1,\ldots \ \ i=1,2,\ldots$$
	
	
	\item $X \sim Poisson(\lambda)$
\end{enumerate}

\newpage
\section*{08/04}

$(\Omega,\mathscr{A},P) \overset{X}{\longrightarrow}(\mathbb{R},\mathscr{B}(\mathbb{R}),P_x) \Leftrightarrow P(A)=P(X^{-1}(B))=P(X\in B)\ f.d.a $ 

\begin{enumerate}
	\item  Variável aleatória discreta, $P(X\in \{x_1,x_2,\ldots\})=1,\forall B in \mathscr{B}(\mathbb{R}) \,$\\
$P(X\in B)= \sum\limits_{i: x_i\in B}P_X(x_i), P_x $ $P_X(x_i), \forall x_i $ é função de probabilidade $P(X=x_i)$

\begin{itemize}
	\item $ B=(-\infty,x]$, $P(X\le x)=\sum\limits_{i:x_i\le x} P(X=x_i)$ f.d.a
	\item $B=[a,b]$,  $a,b\in\mathbb R$
	$$P(a\le X\le b)=\sum\limits_{i:x_i=a}^{b}P(X=x_i) $$
\end{itemize}

\item Variável aleatória Absolutamente Contínua (a.c.)\\
X é a.c. se existir $f:\mathbb R \longrightarrow \mathbb R$\\
Tais que $f(x)\ge0, \forall x\in \mathbb R$ e $\int\limits_{\mathbb R} f(x)dx=1$,\\
chamada de função de densidade de probabilidade (fdp).\\
Neste caso,
$$P(X\in B)=\int\limits_{[X\in B]=\{\omega \in \Omega: X(\omega)=x\in B \}=X^{-1}(B)} f(x)dx  $$
\begin{itemize}
	\item $(-\infty,x], \forall x \in \mathbb R$
	$$P(X\le x) = \int\limits_{X^{-1}\bigg((-\infty,x]\bigg) = \bigg[X\in (-\infty,x]\bigg]=[X\le x]} $$
	
	\item $B=[a,b]$
	$$P(a\le x\le b)\int\limits_{a}^{b} f(x)dx$$
\end{itemize}
\end{enumerate}

\newpage
\underline{Exemplos}:
\begin{itemize}
	\item
$$X\sim Exp(\beta) \Leftrightarrow f(x)=\begin{cases}
\beta e^{-\beta x}, \ \ x\ge 0\\
0, \ \ \ \ \ \ \ \  x<0
\end{cases} $$
\item 

$$X\sim Gamma(\alpha,\beta) \Leftrightarrow f(x)=\begin{cases}
\frac{\beta^\alpha x^{\alpha-1}e^{-\beta x}}{\Gamma(\alpha)}, \ \ x\ge 0\\
0, \ \ \ \ \ \ \ \ \ \ \ \ \ \  \   x<0
\end{cases} $$
$$\Gamma(\alpha)= \int\limits_{0}^\infty x^{\alpha-1}e^{-x}dx, \alpha>0 $$
$\begin{cases}
\Gamma(\alpha+1)=\alpha\Gamma(\alpha)\\
\alpha=n\in \mathbb N, \ \ \ \Gamma(n)=n!\\
\Gamma\bigg( \frac{1}{2}\bigg) = \sqrt \pi\\
\Gamma(1)=1
\end{cases}
$\\
$\int\limits_{0}^{\infty} x^{a-1} e^{- bx^c}dx=\frac{\Gamma (\frac{a}{c} )}{c\cdot b^{\frac{a}{c}}} $
\item $X\sim N(\mu,\sigma^2)$
\item $X\sim W(\alpha,\beta) $
\item $X\sim ogN(\mu,\sigma^2) $
\item $X\sim logistica (\alpha)$
\item $X\sim Cauchy(\alpha)$ 
\end{itemize}

\newpage

\subsection{Função de variável aleatória}
Seja $X:\Omega \rightarrow \mathbb R $ variável aleatória e $g: \mathbb R \longrightarrow \mathbb R$\\
\underline{Borel Mensurável}. Então $Y:\Omega \rightarrow \mathbb R$, $Y=g(x)$ é variável aleatória.

$$\Omega,\mathscr{A} \overset{X}{\longrightarrow}  \mathbb R , \mathscr{B}(\mathbb{R})  \overset{g}{\longrightarrow} \mathbb R,\mathscr{B}(\mathbb{R})  $$

Y é variável aleatória se $\forall B \in \mathscr{B}(\mathbb{R}), \ Y^{-1}(B)\in \mathscr A$,
$$Y^{-1}(B)= X^{-1}\bigg( g^{-1}(B)\bigg)=[y\in B]= [X\in g^{-1}(B)] $$
$$\Leftrightarrow P(y\in B = P\bigg( X\in g^{-1}(B)\bigg) $$
\underline{Casos:}
	\begin{enumerate}[label=\alph*)]
		\item X discreta, $P(X\in \{x_1,x_2,\ldots\})=1\Rightarrow$ Y discreta
		$$\Omega \overset{X}{\longrightarrow} \{x_1,x_2,\ldots\} \overset{g}{\longrightarrow} \{y_1,y_2,\ldots\} $$
		$$Y=g(X) $$
		Caracterizada por
		$$P(Y=y)=P(Y\in \{y\}) = \sum\limits_{=\underset{x_i:g(x_i)=y}{i:\{x_i \in g^{-1}(\{y\}) \}}}^{\infty}P(X=x_i), \ \ B=\{y\} $$
		
		\item 
		
		X a.c. com fdp $f:\mathbb R \longrightarrow \mathbb R \Rightarrow Y=g(x)$?\\
		Temos que, $\forall B\in \mathscr{B}(\mathbb{R}) $, $B=(-\infty,y]$
		$$P(Y\le y) = \int\limits_{=\underset{[x:g(x)\in (-\infty,x]}{[X \in g^{-1}((-\infty,x])]] }}^{\infty}f(x)dx$$ 
		\underline{Y discreta}
		$$P(Y=y)=\int\limits_{[x:g(x)=y]}f(x)dx $$
		\underline{Y a.c.}\\
		fdp:  $F'_Y(y) = f_Y(y) \forall y $\\
		Em particular, se g for inversível:
		$$F_Y(y)=P(Y\le y) $$
		$$=P\bigg( g(X) \le y \bigg) $$
		$$=P\bigg(X \le g^{-1}(y) \bigg)$$
		$$=F_X\bigg( g^{-1}(y)\bigg) \Leftrightarrow F'_y(y) = f_x\bigg(g^{-1}(y)\bigg)\bigg|\frac{d}{dy}\bigg(g^{-1}(y)\bigg) \bigg|$$
	\end{enumerate}
\newpage
\underline{Exemplo}:\\
\begin{enumerate}[label=\arabic*)]
	\item Considere $X\sim B(n,0.5)$: o número de caras que aparecem em n lançamentos de uma moeda honesta.\\
	Um jogador lança a moeda e 
	$\begin{cases}
		ganha \ 1, se\ sair\ cara\\
		perde\ 1, se\ sair\ coroa
	\end{cases}$  \\
	\\
	Determine a distribuição do ganho do jogador\\
	\\
	\underline{Solução:}\\
	Y = Ganho do jogador após n lançamentos
 $$Y=\underbrace{X}_{Caras}-\underbrace{(n-X)}_{Coroas} = 2X-n$$
 $$\Omega \overset{X}{\longrightarrow}  \{0,1,2,\ldots,\}\overset{g}{\longrightarrow} \{-n,2-n,\ldots,n\} $$
 $$Y=g(x) $$
 $$\forall y\in \mathbb R, \ \ P(Y=y)= \sum\limits_{x_i:g(x_i)=y} P(X=x_i)$$
 $$ g(x)=2x-n =y\Rightarrow x = \frac{y+n}{2}$$
  $$\forall y\in \mathbb R, \ \ P(Y=y)= \sum\limits_{\{x_i:g(x_i)=y\}} P(X=x_i)= P(X=\frac{y+n}{2})= 
  \binom{n}{\frac{y+n}{2}} \bigg(\frac{1}{2}\bigg)^n$$
  ou 
  $P(Y=y)= P(2x-n=y)=P(X=\frac{y+n}{2}) $
  	pois g é inversível 
  	
  	\item 
  	$X\sim Binom(3,\frac{1}{2}) $ e $Y=|X-1|$\\
  	Determine a distribuição de Y\\
  	\underline{Sol}:
  	$$\Omega \overset{X}{\longrightarrow}  \{0,1,2,\ldots\}\overset{g}{\longrightarrow} \{0,1,2\} $$
  $$Y=g(x) $$
  $$P(Y=0)=P(X=1) $$
    $$\forall y\in \mathbb R, \ \ P(Y=1)= \sum\limits_{\{x_i:g(x_i)=1\}} P(X=0)+P(X=2)$$
    $$P(Y=2)=P(X=3)$$
    \newpage
    \item $$X\sim U[0,1],  \ \ \ g(x)=\begin{cases}
  1,  \ \ \ x\le \frac{1}{2}\\
  0,\ \ \  x>\frac{1}{2}
    \end{cases} $$
    e $Y=g(x) $, Determine a distribuição de Y.\\
    \underline{Sol}:
    $$\Omega \overset{X}{\longrightarrow} [0,1] \overset{g}{\longrightarrow} \{0,1\}  $$
    $$P(Y=0)= \int\limits_{x:g(x)=0}f(x) dx = \int\limits_{0.5}^1 1 dx= \frac{1}{2}$$
        $$P(Y=1)= \int\limits_{x:g(x)=0}f(x) dx = \int\limits_{0}^{0.5} 1dx = \frac{1}{2}$$
        
        \item $X\sim U[0,1]$ e $g(x)=-log(x)$, $Y=g(x)$\\
        Determine a distribuição de Y\\
        \\
        \underline{Sol}:
        $$\Omega \overset{x}{\longrightarrow} [0,1] \overset{g}{\longrightarrow} \mathbb R   $$
        $$Y=g(x) $$
       $$f_Y(y)= f_x\bigg(g^{-1}(y)\bigg)\bigg|\frac{d}{dy}\bigg(g^{-1}(y)\bigg) \bigg|$$
       $$-log(x)=y \Leftrightarrow x=e^{-y}=g^{-1}(y) $$
       Assim,
       $$f_Y(y)= f_x\bigg( e^{-y}\bigg)\bigg| -e^{-y} \bigg|, \ \ \ f_X(x)= \begin{cases}
       1, \ \ \ x\in[0,1]\\
       0, \ \ \ c.c.
       \end{cases}$$
       
       $$x\in [0,1]\Leftrightarrow y\ge 0 \Rightarrow f_Y(y)= e^{-y} $$
              $$x\notin [0,1]\Leftrightarrow y< 0 \Rightarrow f_Y(y)= 0 $$
              $$\therefore Y\sim Exp(1)$$
       
\end{enumerate}
\newpage
\section*{Função de Distribuição }

\underline{Definição}: uma função $F:\mathbb R \longrightarrow \mathbb R $ é unfção de distribuição (f.d.), se 
\begin{enumerate}
	\item $\forall x \in \mathbb R$, F é não decrescente;
	$$x<y \Rightarrow F(X)\le F(y) $$
	\item F é contínua a direit.\\
	$\{x_n\}$ é uma sequencia t.q. $x_n \downarrow x\Rightarrow \lim\limits_{n\rightarrow \infty}  F(X_n)=F(x)  $
	\item 
	$\lim\limits_{n\rightarrow \infty} F(x)=1$ e$ \lim\limits_{n\rightarrow -\infty} F(x)=0$
\end{enumerate}
\underline{Tipos}:\\
\begin{enumerate}
	\item F.d Discreta\\
	uma f.d. F é discreta se existir uma função p t.q.
	$P(x_i)\ge 0 $,  $\sum\limits_{i=1}^{\infty}P(x_i)=1$, $x_i\in \mathbb{R}, i\ge 1$\\
	Então,\\
	$\forall x\in \mathbb R$, $F(x) = \sum\limits _{i:x_i\le x}P(x_i)$
	
	\item F.d absolutamente contínua  (a.c)\\
	Uma f.d. é a.c. se existir uma função f t.q. $f(x)\ ge 0, \forall x \in \mathbb R$ e $\int\limits_{\mathbb R}f(x)=1$ \\
	Neste caso,
	$\forall x \in \mathbb R $,
	$$F(x) = \int\limits_{-\infty}^{\infty}f(x)dx $$
	Daí,
	$$F'(x)= f(x) \ f.d.p$$
	\item F.d Mista 
	$$F=F_d+F_c $$
	$F_d $: Parte discreta\\
		$F_c $: Parte contínua, $F_c=F_{a.c}+F_s$\\
		Sendo $F_{ac}$: parte a.c. e $F_s$: parte singular\\
		$F_{ac}$ é t.q. $F_{ac}=f(x)$ f.d.p e $F_s'(x)=0$
		$$\therefore F=F_d+F_{ac}+F_s $$
\end{enumerate}
\newpage
\underline{Passos}:
\begin{enumerate}
	\item Identificar os pontos de descontinuidade 
	$$F_j\bigg( \{x_1,x_2,\ldots \}\bigg) $$
	
	\item Calcular o tamanho do salto em cada $x_i$,
	$$b_{x_i}= F(x_i^+) - F(x_i^-) = F(x_i)-F(x_i^-)$$
	
	\item Parte discreta:
	$$F_d(x)=\sum\limits_{i:x_i\le x} b_{x_i}, \ \forall x\in \mathbb R $$
	\item Parte a.c.
	$$F_{ac}'(x)=f(x)=F'(x), $$
Pois $$F'(x)=\cancelto{0}{F_d'(x)}+F'_{ac}(x)+\cancelto{0}{F'_s(x)} $$	
$$\Rightarrow F_{ac}(x)=\int\limits_{-\infty}^{x} f(t)dt$$

\item Parte singular
$$F_S=F-(F_x+F_{ac}) $$
\end{enumerate}
\newpage
\underline{Exemplo}:\\

Seja $X\sim U[0,1]$ e $Y=min\bigg\{ X,\frac{1}{2}\bigg\}$

\begin{enumerate}[label=\alph*)]
	\item Determine a distribuição de Y 
	\item Decomponha a distribuição em parte discreta e contínua.
	
\end{enumerate}
\underline{Sol}:\\
\\
\begin{enumerate}[label=\alph*)]
	\item 

 $X\sim U[0,1]\Rightarrow F_X(x)=
 \begin{cases}
 0,\ \ \ x<0\\
 x, \ \ \ 0\le x \le 1\\
 1,\ \ \ x\ge 1
 \end{cases}
 $
e

$$Y=\begin{cases}
X , \ \ \ X\le \frac{1}{2}\\
\frac{1}{2}, \ \ \ X>\frac{1}{2}
\end{cases} $$

$$\Rightarrow \forall y \in \mathbb R,$$
$$P\bigg(Y\le y,X\le \frac{1}{2}\bigg) +P\bigg(Y\le y,X> \frac{1}{2}\bigg)$$
$$= P\bigg(X=Y\le y , X\le \frac{1}{2}\bigg) +P\bigg(Y=\frac{1}{2} ,Y\le y, X> \frac{1}{2}\bigg)$$

$$F_Y(y)=\begin{cases}
0, \ \ \ \ \ \ \ y<0\\
y, \ \ \ \ \ \ \ 0\le y<\frac{1}{2}\\
\underbrace{P\bigg(X\le \frac{1}{2}\bigg)+P\bigg(X> \frac{1}{2}\bigg)}_{=1}, \ \ \ \ \ \ \ y\ge \frac{1}{2}\\
\end{cases} $$

\item 

\begin{enumerate}[label=\roman*)]
	\item  Descontinudade = $\{y_1=\frac{1}{2}\}$
	\item Tamanho do salto:\\
	$$b_{x_1}=F(y_1)-F(y_1^-) $$
	$$=F\bigg(\frac{1}{2}\bigg)- F\bigg(\frac{1}{2}^{-1}\bigg) $$
	$$= 1-\frac{1}{2} =\frac{1}{2} $$
	
	\item $\forall x \in \mathbb R $
	$$F_y(d)= \begin{cases}
	0, \ \ \ \ y\le \frac{1}{2}\\
	\frac{1}{2} \ \ \ \ y \ge \frac{1}{2}

	
	\end{cases} $$
	
	\item $$F'_Y(y)=\begin{cases}
	0, \ \ \ \ \ \ \ y<0\\
	1, \ \ \ \ \ \ \ 0\le y<\frac{1}{2}\\
0,\ \ \ \ \ \ \  y\ge \frac{1}{2}\\
	\end{cases} 
	=\begin{cases}
	1, \ \ \ 0<y<\frac{1}{2}\\
	0, \ \ \ c.c
	\end{cases}=f(x)
	$$
	
	$$\Rightarrow
F_{ac}(y)=\begin{cases}
		0, \ \ \ \ \ \ \ y<0\\
		\int\limits_{0}^{y}1 dy \ \ \ \ \ \ \ 0\le y<\frac{1}{2}\\
	\int\limits_{0}^{\frac{1}{2}}1 dy=\frac{1}{2}, \ \ \ \ \ \ \ y\ge \frac{1}{2}\\
	\end{cases} 
	 $$
	 $$\therefore F_{ac}(y) = \begin{cases}
	 0, \ \ \ \ y<0\\
	 y, \ \ \ \ 0\le y < \frac{1}{2}\\
	 \frac{1}{2}, \ \ \ y>\frac{1}{2}
	 \end{cases} $$
	 
	 \item $$F_s(x) = F(y)- \bigg( F_d(y)- F_{ac}(y)\bigg) $$
	 
	 $$F_d(y)- F_{ac}(y)=\begin{cases}
	 0, \ \ \ \ \ \ \ y<0\\
y  \ \ \ \ \ \ \ 0\le y<\frac{1}{2}\\
	 1 , \ \ \ \ \ \ \ y\ge \frac{1}{2}\\
	 \end{cases} =F(y) \Rightarrow F_S(y)\equiv 0 $$
	 \end{enumerate}


\end{enumerate}
\newpage
\underline{f.d. de variável aleatória}:\\
X v.a.
$$\rightarrow F(x)= P_X\bigg((-\infty,x]\bigg), \ \ \ \forall x \in \mathbb R \ \ = \ P(X\le x)  \ \ \ f.d $$

\underline{Prova}:\\
\begin{enumerate}[label=\roman*)]
	\item $x<y \Rightarrow (-\infty,x]\subset (-\infty,x]$\\
	 $ P\bigg((-\infty,x]\bigg)\le P\bigg((-\infty,x]\bigg)$\\
	 $$\Rightarrow F(x)\le F(y) $$
	  \item $\{x_n\} $, $x_n \downarrow x $
	  $$(-\infty,x_1]\subset (-\infty,x_2]\subset\ldots \bigcup\limits_{n\ge 1}A_i(-\infty,x_n]= (-\infty,x] $$
	  $$\lim\limits_{n\rightarrow \infty} F(x_n)= \lim\limits_{n\rightarrow \infty} P_X\bigg((-\infty,x] \bigg) $$
	  Como $\{(-\infty,x]\}_{n\ge 1}$ é uma sequencia decrescente de eventos, pela continuidade de probabilidade.
	  $$\lim\limits_{n\rightarrow \infty} P_X\bigg((-\infty,x_n] \bigg) = P_X\bigg((-\infty,x] \bigg)=F(x) $$
	  $$\Rightarrow \lim\limits_{n\rightarrow \infty} F(x_n)=F(x) $$
	  
	  \item $$x_n\downarrow-\infty \Rightarrow \lim\limits_{n\rightarrow \infty} F(x_n)=0 $$
	  e 
	  $$x_n\uparrow \infty \Rightarrow \lim\limits_{n\rightarrow \infty}  F(x)-1 $$

\end{enumerate}
\newpage

\section{22/04}

\underline{Vetores Aleatórios}\\
\\

Seja$ X_1,\ldots,X_n$ variaveis aleatorias definidas sob $(\Omega,\mathscr{A},P)$\\
Então, $\bm X=(X_1,\ldots,X_n): \underset{\omega \longrightarrow X(\omega)=\bm x}{\Omega\longrightarrow \mathbb R^n}$ é um vetor aleatório se:\\
$\forall \bm x = (x_1,\ldots,x_n) \in \mathbb R^n$, $A = \underbrace{\bm X^{-1}\bigg((-\infty,\bm x]\bigg)}_{ \{\omega \in \Omega: X(\omega)\in (-\infty,\bm x]\}}
 \in \mathscr A = \sigma(X_1,\ldots,X_n)$,
 \\
 Sendo \\
 $\underbrace{(-\infty,\bm x]}_{\text{Retângulo de } \ \mathbb R^n} = (-\infty, x_1]\times \ldots \times (-\infty,x_n]\in \mathbb R^n $\\
 Então:\\
 $$\underbrace{P_{\bm X}(\bm x) = P\bigg(\bm X^{-1}\bigg((-\infty,\bm x)\bigg)\bigg)= P\bigg(\bm X\le \bm x\bigg)}_
 {\text{Função de distribuição Acumuladada de } \ \bm X}$$
$$=P(X_1\le x_1,\ldots, X_n\le x_n) $$
$$= \underbrace{P\bigg (X_1,\ldots,X_n)\in  (-\infty, x_1]\times \ldots \times (-\infty,x_n]\bigg)}_
{\bigg\{\omega:\bigg(X_1(\omega),\ldots,X_n(\omega)\bigg)\in (-\infty,x_1]\times\ldots\times(-\infty,x_n] \bigg\}} $$
\underline{Casos Particulares}\\
\\
\begin{enumerate}
	\item Vetor Aleatório discreto
	$$\bm X:\Omega  \longrightarrow\{\bm x_1,\bm x_2\,\ldots \} $$
	Tal que:
	$$P_X: \ \sigma(X_1,\ldots, X_2) \longrightarrow[0,1]$$
	Satisfaz:
	\begin{enumerate}[label=\roman*)]
		\item $\forall \bm x,  \ \ P_{\bm X}(\bm x)= P(\bm X = \bm x)\ge 0 $
		\item $\sum\limits_{\bm x}P(\bm X = \bm x)=1$ \\
		$\Leftrightarrow  P(\bm X=\bm x)=P(X_1=x_1,\ldots,X_n=x_n)\ge 0  $\\
		e\\
	 $\sum\limits_{i_n}\cdots \sum\limits_{i_1}P\bigg(X_1=x_{i1},\ldots,X_n=x_{in}\bigg)=1$\\
	 $\bm x_i=(x_{i1},\ldots x_{in})$.\\
	  Neste caso, $P_X$ é chamada de \underline{Função de probabilidade conjunta de}
	 $X_1,\ldots,X_n $.\\
	 \\
	 Assim,\\
	 fda: $P_{\bm x}\bigg((-\infty,\bm x]\bigg) = P\bigg(\bm X^{-1}(\infty,\bm x]  \bigg)=P\bigg( \bm X\le \bm x\bigg)$\\
	 $$\Rightarrow F_{\bm X}(\bm x) = P(\bm X\le \bm x)= \sum\limits_{i_n\le x_n}\cdots \sum\limits_{i_1\le x_1}
	 P\bigg(X_1=x_{i1},\ldots,X_n=x_{in}\bigg)
	  $$
	  \underline{Exemplo} (E18):\\
	  $X:\Omega \rightarrow \{1,2,3\} $: número da primeira bola retirada\\
	  $Y:\Omega \rightarrow \{1,2,3\} $: número da segunda bola retirada\\
	  $\bm X = (X,Y): \underset{\omega \longrightarrow \bm X(\omega)}{\Omega \longrightarrow} I \subset \{1,2,3\}\times \{1,2,3\}$\\
	  $I=\bigg\{(1,2),(1,3),(2,1),(2,3),(3,1),(3,2)   \bigg\}$
	  \\
	  $\bm X(\omega)=(X(\omega),Y(\omega) $\\
	  \begin{enumerate}[label=\alph*)]
	  	\item $P(X=x,Y=y)=\frac{1}{A_{3,2}}=\frac{1}{6},\forall(x,y)$\\
	  	Satisfaz:\\
	  	$P(X=x.Y=y)=\frac{1}{6}>0$\\
	  	e\\
	  	$\sum\limits_{(x.y)\in I}P(X=x,Y=y)=1$
	  	\item $P(X<Y)=\frac{3}{6}=\frac{1}{2}$
	  \end{enumerate}
	  
	\end{enumerate}

\item \underline{Vetor aleatório absolutamente contínuo}\\
$\bm X=(X_1\ldots,X_n):\Omega \longrightarrow \mathbb R$ é absolutamente contínuo se existir uma função, chamda de f.d.p conjunta,
$$f:\mathbb R^n \longrightarrow \mathbb R $$
\begin{enumerate}[label=\roman*)]
	\item $f(\bm x)\ge 0, \ \ \forall \bm x=(x_1,\ldots,x_n)\in \mathbb R$
	\item $\int\limits_{\mathbb R^n} f(\bm x)d\bm x=\int\limits_{\mathbb R}\cdots \int\limits_{\mathbb R}f(x_1,\ldots,x_n)dx_1\ldots dx_n = 1$\\
	a f.d.a conjunta, é dada por:
	$$F_{\bm X}(\bm x) = P\bigg(\bm X^{-1}\bigg((-\infty,x]\bigg)\bigg)=P(\bm X\le \bm x) $$
	$$= \int\limits_{-\infty}^{\bm x}f(\bm t)d\bm t = \int\limits_{-\infty}^{x_1}\cdots \int\limits_{-\infty}^{x_n}f(t_1,\ldots,t_n)dt_1,\ldots,dt_n $$
	Tem-se também que:\\
	$$\frac{\partial}{\partial \bm x}F_{\bm X}(\bm x)= \frac{\partial F(x_1,\ldots,x_n)}{\partial x_1\ldots \partial x_n } =f(x_1,\ldots,x_n)  $$
\end{enumerate}
\underline{Exemplos:}
\begin{enumerate}
	\item $\bm X=(X,Y)$\\
	$P\bigg((X,Y)\in (a_1,b_1]\times(a_2,b_2] \bigg)= P(a_1< X\le b_1,a_2< Y\le b_2)$ \\
	$\begin{cases}
		\sum\limits_{a_2\le y \le b_2}\sum\limits_{a_1\le x \le b_1}P(X=x,Y=y), \ \ \ \ \bm X \ discreto\\
\int\limits_{a_2}^{b_2}\int\limits_{a_2}^{b_2}f(x,y)dxdy, \ \ \ \ \ \bm X \ absolutamente \ continuo
	\end{cases}$
\end{enumerate}
\underline{Distribuição Marginal}:\\
Seja $\bm X=(X_1,\ldots,X_n )$ um vetor aleatório com f.d.a $F_{\bm X}(x_1,\ldots,x_n)$\\
Então:
$$\lim\limits_{x_k\rightarrow \infty}F_{\bm X}(x_1,\ldots,x_{k-1},x_k,x_{k+1},\ldots,x_n)$$
$$ =
\lim\limits_{n\rightarrow \infty} P\bigg(
X_1^{-1}\bigg((-\infty,x_1]\bigg),\ldots, \underbrace{X_k^{-1}\bigg((-\infty,\cancelto{\infty}{x_k}]\bigg)}_{\Omega},\ldots, X_n^{-1}\bigg((-\infty,x_n]\bigg)
\bigg)$$
$$= \underbrace{F_{X_1\ldots,X_{k-1},X_{k+1},\ldots X_n}(x_1,\ldots,x_{k-1},x_{k+1},\ldots,x_n)}_{\text{Distribuição marginal de ordem n-1}} $$ 
Em particular,
$$\bm X=(X,Y) $$
$$ \underbrace{F_{X,Y}(x,y) = P(X\le  x,Y\le y)}_{\text{f.d.a conjunta de X e Y}} $$
$$\Rightarrow \lim\limits_{x\rightarrow \infty} F_{X,Y}(x,y)=F_Y(y) $$
$$\Rightarrow \lim\limits_{y\rightarrow \infty} F_{X,Y}(x,y)=F_X(x) $$
São distribuições marginais de Y e X respectivamente.
\begin{itemize}
	\item Distribuição marginais $\not\Rightarrow$ Determina modelo conjunto
	\item Modelo conjunto $\Rightarrow$ Distribuição marginais
\end{itemize}
\underline{Casos Particulares}\\

\begin{itemize}
	\item  $\bm X=(X_1,\ldots,X_n)$ Discreto\\
	$\Rightarrow P(X_1=x_1,\ldots, X_n=x_n) $ f.p conjunta \\
	As probabilidade marginais,
	$$P(X_1=x_1,\ldots,X_{n-1}=x_{n-1})
	=
	P\bigg( X_1^{-1}(\{x_1\}),\ldots, X_{n-1}^{-1}(\{x_{n-1}\})\bigg)
	 $$

$$P(X=x,Y=y)$$
$$ \Rightarrow  P(X=x)=\sum\limits_{y}P(X=x,Y=y)$$
$$ \Rightarrow  P(Y=y)=\sum\limits_{x}P(X=x,Y=y)$$
São funções de probabilidade marginal de X e Y, respectivamente, Pois:\\
$$P(X=x)=P\bigg(X^{-1}(\{x\}) \bigg) = P\bigg(X^{-1}(\{x\}),\Omega \bigg) 
=
 P\bigg(X^{-1}(\{x\}),Y^{-1}\bigg(
\bigcup_{y_i}\{y_i\}  \bigg) \bigg)
$$
$$
=\sum\limits_{y_i}P\bigg(
X^{-1}(\{x \}),Y^{-1}(\{y_i\})\bigg)
=\sum\limits_{y_i}P\bigg(X=x,Y=y
\bigg)
$$
\item $\bm X = (X_1,\ldots,X_n)$ absolutamente continuo $\Leftrightarrow$ fdp conjunta $f_{\bm X}(\bm x)$\\
As densidade marginais são:
$$f_{X_1,\ldots,X_{n-1}}(x_1,\ldots,x_{n-1})=\int\limits_{\mathbb R}f(t_1,\ldots,t_n)dt_n $$
n=2:\\
$f(x,y)$ f.d. conjunta e 
$$f_X(x)= \int\limits_{\mathbb R}f(x,y)dy, \ \ \ \ f_Y(y)= \int\limits_{\mathbb R}f(x,y)dx $$
São as densidades marginais de X e Y, respectivamente. 
\end{itemize}
\end{enumerate}
\underline{Definição:} As variaveis aleatorias $X_1,\ldots,X_n$ são independentes, se:
$$F_{X_1,\ldots,X_n}(x_1,\ldots,x_n)=F_{X_1}(x_1)\cdots F_{X_n}(x_n), \ \ \ \forall
\bm x\in \mathbb R^n$$
Em particular:\\
\begin{itemize}
	\item $X_1,\ldots,Xn $ discretas  são independentes, se:\\
	 $$P(X_1=x_1,\ldots,X_n=x_n)=P(X_1=x_1)\cdots P(X_n=x_n)$$
	 \item \item $X_1,\ldots,Xn $ abs continuas  são independentes, se:\\
	 $$f_{\bm X}(\bm x)= f_{X_1}(x_1)\cdots f_{X_n}(x_n) $$
\end{itemize}
\newpage
\section{24/04}
$$ 
X\sim N(\mu,\sigma^2)\Leftrightarrow f(x) = \frac{1}{(2\pi)^{\frac{1}{2} }(\sigma^2)^\frac{1}{2}}\exp \bigg\{- \frac{1}{2}\bigg(
\frac{x-\mu}{\sigma}
\bigg)^2\bigg\}$$
$$
=
\frac{1}{(2\pi)^{\frac{1}{2} }(\sigma^2)^\frac{1}{2}}
\exp \bigg\{- \frac{1}{2}\
(x-\mu)'(\sigma^2)^{-1}(x- \mu)
\bigg\}
$$

$$
\bm X = (X_1,\ldots,X_n) \Leftrightarrow f(\bm x) = 
\frac{1}{(2\pi)^{\frac{n}{2} }|\Sigma|^\frac{n}{2}}
\exp \bigg\{- \frac{1}{2}\
(\bm x-\mu)'(\sigma^2)^{-1}(\bm x- \mu)
\bigg\}
$$

$$\bm X \sim N(\mu,\Sigma) $$

$$\bm X = (X,Y)\sim N(\mu,\Sigma), \ \ \ \mu=(\mu_1,\mu_2), \ \ \ \Sigma = \begin{bmatrix}
\sigma_1^2 & \sigma_{21}\\
\sigma_{21} & \sigma_2^2
\end{bmatrix} $$

$$f(x,y) = 
\frac{1}{2\pi \sigma_1\sigma_2(1-\rho_{12})^2}
\exp\bigg\{
\frac{1}{2(1-\rho_{12})^2}\bigg[
\bigg(
\frac{x-\mu_1}{\sigma_1}
\bigg)^2
-
2\rho_{12}
\bigg(
\frac{x-\mu_1}{\sigma_1}
\bigg)
\bigg(
\frac{y-\mu_2}{\sigma_2}
\bigg)
+
\bigg(
\frac{y-\mu_2}{\sigma_2}
\bigg)^2
\bigg]
\bigg\}
 $$
 
 
 $$
 f(x) =\int\limits_{-\infty}^{\infty}  f(x,y)dy= \frac{1}{(2\pi)^{\frac{1}{2} }(\sigma^2)^\frac{1}{2}}\exp \bigg\{- \frac{1}{2}\bigg(
 \frac{x-\mu}{\sigma}
 \bigg)^2\bigg\}$$
$$
 f(y) =\int\limits_{-\infty}^{\infty}  f(x,y)dx= \frac{1}{(2\pi)^{\frac{1}{2} }(\sigma^2)^\frac{1}{2}}\exp \bigg\{- \frac{1}{2}\bigg(
\frac{y-\mu}{\sigma}
\bigg)^2\bigg\}$$

$$\rho_{12}=\frac{\sigma_{12}}{\sigma_1\sigma_2} $$

$$\therefore \bm X\sim N(\mu,\Sigma)\underbrace{\Rightarrow}_{\Leftarrow \ valido,  \ \ \rho=0} X_i \sim N(\mu_i,\sigma^2_i),\ \ i=1,\ldots,n $$

Por outro lado, correlação nula $\not\Rightarrow$ variáveis aleatórios (componentes) são independentes  \\
\\






\newpage
\underline{Função de distribuição multivariada }\\
\underline{Definição:} Uma função de distribuição $F:\mathbb R^n \longrightarrow[0,1]$ é uma função de distribuição n-dimensional, se satisfaz:
\begin{enumerate}[label=\roman*)]
	\item F é não decrescente em cada componente;
	$$\forall x,y,\in \mathbb R, x<y\Rightarrow 
	F(x_1,\ldots,x,\ldots,x_n)\le 	F(x_1,\ldots,y,\ldots,x_n)
	 $$
	 \item F é contínua á direita em cada componente;

Dada $y_n\downarrow y$,
$$\lim\limits_{n\rightarrow \infty} F(x_1,\ldots,y_n,\ldots,x_n)= 	F(x_1,\ldots,y,\ldots,x_n) $$
\item 
$$\lim\limits_{x_i\rightarrow \infty} F(x_1,\ldots,x_k,\ldots,x_n)=1, \ \ \ \ i=1,\ldots,n $$
e
$$\lim\limits_{para \ algum \ x_i\rightarrow- \infty} F(x_1,\ldots,y_n,\ldots,x_n=0)$$
\item  $\forall$ retângulo de $\mathbb R^n$, $(a,b ] =  (a,b ]\times\ldots\times (a_n,b_n ]$, $Vol_F\bigg(
 (a,b ]
\bigg)\ge 0 $

$$Vol_F\bigg(
(a,b ]
\bigg) =  \Delta^{b_n}_{a_n}\ldots\Delta^{b_1}_{a_1}F(t_1,\ldots,t_n)$$
Com
$$\Delta^{b_k}_{a_k}F(t_1,\ldots,t_n)=F(t_1,\ldots,b_k,\ldots,t_n) -F(t_1,\ldots,a_k,\ldots,t_n)  $$
Para n=2:

$$Vol_F\bigg((a,b]\bigg)=Vol_F\bigg(
(a_1,b_1]\times(a_2,b_2]
\bigg)=\Delta^{b_2}_{a_2}\Delta^{b_1}_{a_1}F(t_1,t_2) $$

$$Vol_F\bigg(
(a_1,b_1]\times(a_2,b_2]
\bigg)=\Delta^{b_2}_{a_2}\bigg[
F(b_1,t_2)-F(a_1,t_2)
\bigg] 
$$
$$
=
F(b_1,b_2)-F(b_1,a_2)-F(a_1,b_2)+F(a_1,a_2)\ge 0
$$
\newpage
\underline{Proposição:}\\
\\
Seja $\bm X=(X_1,\ldots,X_n):\Omega \longrightarrow \mathbb R^n$ um vetor aleatório. Então 
$F_{\bm X} :\mathbb R^n \longrightarrow [0,1]$ definida por:
$$F_{\bm X}(\bm x)= P\bigg(\bm X^{-1}\bigg((-\infty,\bm x]\bigg) \bigg), \forall \bm x\in \mathbb R $$
$$=P\bigg(\bm X \in (-\infty,\bm x]\bigg) =P(\bm X\le \bm x)  $$
É função de distribuição n-variada.\\
\\
\underline{Prova}:
$$F_{\bm X}(\bm x)=F_{\bm X}(x_1,\ldots,x_n)
=P(
X_1\in (-\infty,x_1],
\ldots,
X_n\in (-\infty,x_n]
) $$
Provar seguindo prova para o caso n=1

$$Vol_F\bigg(
(a,b]
\bigg)=
\Delta^{bn}_{an}\ldots\Delta_{a_1}^{b_1}F(t_1,\ldots,t_n)=P(a_1<X_1\le b_1,\ldots,a_n<X_n\le b_n) \ge 0$$
$$Vol_F\bigg(
(a,b]
\bigg)
=
P(a_1<X_1\le b_1,a_2<X_2\le b_2)
 $$
 
 $$
 =P(X_1\le b_1,a_2,X_2\le b_2) - P(X_1\le a_1,a_2<X_2\le b_2)
 $$
 



\end{enumerate}
	
	
	\newpage
	\section{29/04}
	\underline{Função de Vetor Aleatório}
	Seja $\bm X = (X_1,\ldots X_n)$ um vetor aleatório definido em $(\Omega,\mathscr{A},P)$ \\
	Então: 
	$$\underset{\bm X^{-1}\bigg(g^{-1}(B)\bigg)\in \mathscr A }{\Omega} \overset{\bm X}{\longrightarrow}  
	\underset{g^{-1}(B)\in \mathscr{B}(\mathbb{R}^n) }{\mathbb R^n}
	\overset{g}{\longrightarrow} 
	\underset{B \in \mathscr{B}(\mathbb{R}) }{\mathbb R}
	 $$
	$$Z=g(\bm X) $$
	Se $g:\mathbb R^n \rightarrow \mathbb R$ é Borel mensurável,\\
	\\
	$Z=g(\bm X): \Omega\longrightarrow\mathbb R$ é uma variável aleatória, Então:\\
	\\
	$$\forall B \in \mathscr{B}(\mathbb{R}),  $$
	$$Z^{-1}(B)=[Z\in B] $$
$$=\bm X^{-1}\bigg(g^{-1}(B)\bigg) $$
$$=[\bm X \in g^{-1}(B)] $$
Assim, 
$$P\bigg(Z^{-1}(B)\bigg)=P(Z\in B)=P\bigg(\bm X \in g^{-1}(B) \bigg) $$
$\forall B\in \mathscr{B}(\mathbb{R}) $ é f.d.a de Z\\
\\
\underline{Casos Particulares}\\
\begin{enumerate}[label=\arabic*)]
	\item $\bm X$ é discreto $\Rightarrow Z=g(\bm X)$ discreto.
	
	$$P(Z\in B) = P\bigg(\bm X \in g^{-1}(B)\bigg) $$
	$$=\sum\limits_{\bm x_i: g(\bm x_i)\in B} P(\bm X=\bm x_i) $$
	Também:\\
	\\
	f.p. :
	$$P(Z=z)=\sum\limits_{\bm x_i=g(\bm x_i)=z} P(\bm X = \bm x_i) $$
	
	\item $\bm X$ absolutamente contínuo $   \Rightarrow Z= g(\bm X)\begin{cases}
		Discreta\\
		Continua
	\end{cases}$
	$$
	P(Z \in B) = P\bigg(\bm X \in g^{-1}(B)\bigg)	$$
	$$= \int\limits_{\bm x: g(\bm x\in B} f_{\bm X}(\bm x)dx \overset{\bm X =(X,Y)}{=}
	\int\limits_{(x,y):g(x,y)\in B}f_{(X,Y)}(x,y)dxdy
	$$
	

	
\end{enumerate}
	\underline{Método do Jacobiano} (Teorema da função Implícita)\\
	\\
	$
	h:\begin{cases}
	\mu = x\\
	v=g(x,y), \ \ \ \ v=z
	\end{cases}
	$
	\\
	\\
	Se as derivadas parciais de $\mu$ e $v$ existem, são contínuas e 
	$$
	|J(x,y)| = \bigg|\begin{pmatrix}
	\frac{\partial\mu}{\partial x} & \frac{\partial\mu}{\partial y}\\
	\frac{\partial v}{\partial x}   & \frac{\partial v}{\partial y} 
	\end{pmatrix}\bigg|
	$$
	Então pelo Teorema da função Implícita existe $h^{-1}$ e:
	
	$$
	h^{-1}:\begin{cases}
	x=\mu\\
	y=y(\mu,v)
	\end{cases}
	\Rightarrow
	\begin{cases}
	dx=d\mu\\
	dy = dh^{-1}(\mu,v)\\
	dy=\frac{1}{J(\mu,v)}dv
	\end{cases}
	 $$
	Logo,
	$$P(Z\le z)= 
	\iint\limits_{\{(x,y): g(x,y)\le z \}}
	f_{X,Y}(x,y)dxdy
	 $$
	 $$
	 =\int\limits_{-\infty}^{z}\bigg[\int\limits_{-\infty}^{\infty}f_{X,Y}\bigg(\mu,y(\mu,v\bigg)d\mu\bigg] \bigg|\frac{1}{J(\mu,v)}\bigg|dv
	 $$
	 
	 $$
	\therefore F_Z(z)=P(Z\le z)	= \int\limits_{-\infty}^{z}\bigg[\int\limits_{-\infty}^{\infty}f_{X,Y}\bigg(\mu,y(\mu,v\bigg)d\mu \bigg|\frac{1}{J(\mu,v)}\bigg|\bigg]dv
	 $$
	 
	 $$
	 \Rightarrow F_Z'(v) = \int\limits_{-\infty}^{\infty}f_{X,Y}\bigg(\mu,y(\mu,v)\bigg)
	 \frac{1}{|J(\mu,v)|}d\mu=f_Z(v), \indent v=z=g(x,y)
	 $$
	
	\newpage
	
	\underline{Exemplo}:\\
	\\
	Sejam $X,Y$ i.i.d $U(0,1)$. Determine a densidade de $Z=\frac{X}{Y}$\\
	\\
	\underline{Solução}:\\
	Defina:\\
	\\
	$h:
	\begin{cases}
		u=x\\
		v=\frac{x}{y}, \indent v=z
	\end{cases} 
	$\\
	\\
	$J(x,y)=\begin{pmatrix}
	1 & 0\\
	\frac{1}{y} & \frac{-x}{y^2}
	\end{pmatrix}
	,
	\indent |J(x,y)| = \frac{-x}{y^2}\ne 0
	$
	\\
	\\
	$
	h^{-1}:\begin{cases}
	x=u
	y=\frac{u}{v}
	\end{cases}
	$\\
	
	$
	\frac{1}{J(\mu,v)}=\begin{pmatrix}
	1 & 0\\
	\frac{1}{v} & -\frac{u}{v^2}
	\end{pmatrix} = -\frac{u}{v^2}
	$
	\\
	$$\Rightarrow f_Z(v)= \int\limits_{-\infty}^{\infty}f_{X,Y}\bigg(u,\frac{u}{v}\bigg)  \frac{|u|}{v^2}du$$
	$$=\int\limits_{-\infty}^{\infty}f_X(u)\cdot f_Y\bigg(\frac{u}{v}\bigg)   \frac{|u|}{v^2}du$$
	$$f_Y\bigg(\frac{u}{v}\bigg) = \begin{cases}
	1, \indent 0<\frac{u}{v}<1 \Leftrightarrow 0<u<v\\
	0, \indent c.c.
	\end{cases} $$
	
	$$v<0 \Rightarrow f_Z(v)=0 $$
	
	$$0<v<1\Rightarrow f_Z(v) = \int\limits_{-0}^{v} 1\cdot 1 \frac{u}{v^2} du = \frac{1}{v^2}\bigg(\frac{u^2}{2}\bigg)\bigg|_0^v=\frac{1}{2}$$
	
	$$v\ge 1 \Rightarrow f_Z(v) = \int\limits_{0}^{1} 1\cdot 1 \frac{u}{v^2}du = \frac{1}{2v^2}$$
	
	$$\therefore f_Z(v)=\begin{cases}
	0,\indent v<0\\
	\frac{1}{2 } \indent 0\le v<1\\
	\frac{1}{2v^2},\indent v\ge 1
	\end{cases} $$
	\newpage
	\underline{Exercício}:
	$$X\sim N(0,1),\indent Y\sim \chi^2_{(n)} $$
	Determina a densidade de $Z=\frac{X}{\sqrt{\frac{Y}{n}}}$ 
	
	
	
	\newpage
	
	(X,Y) vetor aleatório $g(X,Y)=Z$, $g:\mathbb R^2\rightarrow \mathbb R$
	
	$$f_Z(v)= \int\limits_{\mathbb R} f_{X,Y}(u,y(u,v))\bigg|\frac{1}{J(u,v)}\bigg|du $$
	$g:\begin{cases}
	u=x\\
	v=g(u,y)
	\end{cases}
	\Rightarrow det\bigg(J(x,y)\bigg)\ne 0 \bigg(J_h(u,v)=\frac{1}{J(u,v)}\bigg)
	$
	
	$$\exists g^{1}h:\begin{cases}
	x=u\\
	y=y(u,v)
	\end{cases} 
	\Rightarrow 
	\frac{1}{J(u,v)}=det\begin{pmatrix}
	\frac{\partial x}{\partial u}& \frac{\partial x}{\partial v}  \\
		\frac{\partial y}{\partial u}& \frac{\partial y}{\partial v}  
	\end{pmatrix}=J_h(u,v)
	$$
	\newpage
	\underline{Vetor aleatório de vetor aleatório}
	
	$$\Omega  \overset{\bm X}{\longrightarrow}  
\underset{g^{-1}(B)\in \mathscr{B}(\mathbb{R}^n) }{\mathbb R^n}
\overset{\bm g}{\longrightarrow} 
\underset{B \in \mathscr{B}(\mathbb{R}) }{\mathbb R^n}
$$
	$$\bm Z = g(\bm X), \indent \bm Z = g(\bm X)= \bigg(
	g_1(X_1),\ldots,g_n( X_n)
	\bigg)$$
	
	\begin{enumerate}
		\item \underline{$\bm X$ abs. cont. e g inversível}
		Defina
		$$ g:
		\begin{cases}
		z_1=g_1(x_1,\ldots,x_n)\\
		\vdots\\
		z_n = g_n(x_1,\ldots,x_n)
		\end{cases}
		$$
		Se $$
		det(J(x_1,\ldots,x_n))=det\begin{pmatrix}
			\frac{\partial x}{\partial u}&\ldots &\frac{\partial x}{\partial v}  \\
			\vdots & &\vdots\\
			\frac{\partial y}{\partial u}&\ldots & \frac{\partial y}{\partial v}  
			\end{pmatrix}\ne 0$$
			e as derivadas são contínuas, então existe $g^{-1}$,
			$$h=g^{-1}:\begin{cases}
			x_1=h_1(z_1,\ldots,z_n)\\
			\vdots\\
						x_n=h_n(z_1,\ldots,z_n)\\
			
			\end{cases} $$
			e
$$J_h(z_1,\ldots,z_n)=
det\begin{pmatrix}
\frac{\partial x}{\partial u}&\ldots &\frac{\partial x}{\partial v}  \\
\vdots & &\vdots\\
\frac{\partial y}{\partial u}&\ldots & \frac{\partial y}{\partial v}  
\end{pmatrix}
 $$
 Daí
 $$f_{Z'}(z')=f_{X'}\bigg(h_1(z_1,\ldots,z_n),\ldots,h_n(z_1,\ldots,z_n)\bigg)\bigg|
 J_h(z_1,\ldots,z_n)
 \bigg| $$
			Note que para n=2:
			$$f_{Z_1,Z_2}(z_1,z_2)=
			f_{X_1,X_2}\bigg(
			h_1(z_1,z_2),h_2(z_1,z_2)
			\bigg)\bigg|
			J_h(z_1,z_2)
			\bigg|
			 $$
			$$\Rightarrow
			f_{Z_1}(z_1)= \int\limits_{\mathbb R} f_{X_1,X_2}(h_1,h_2)\cdot \bigg|
			J_h(z_1,z_2)
			\bigg|dz_2
			 $$
			 
			 $$
			 			f_{Z_2}(z_2)= \int\limits_{\mathbb R} f_{X_1,X_2}(h_1,h_2)\cdot \bigg|
			 J_h(z_1,z_2)
			 \bigg|dz_1
			 $$
			 
			 Em geral, tem-se que:
			 
			 $$F_{\bm Z}(\bm z)=P(\bm Z\le\bm  z) = \int\cdots\int  f_{\bm X}\bigg(
			 h_1(t_1,\ldots,t_n),\ldots,h_n(t_1,\ldots,t_n)
			 \bigg)
			 \bigg|
			 J_h(t_1,\ldots,t_n)
			 \bigg|d\bm t
			 $$
			 $$\bm t=(t_1,\ldots,t_n),\indent \bm z=(z_1,\ldots,z_n) $$
			 \newpage
			 \underline{Exemplo}:\\
			 \\
			 Considere (X,Y) com fdp conjunta
			 $$f(x,y)= \begin{cases}
			 \frac{1}{x^2y^2}, \ \ \ \ x\ge1,y\ge 1\\
			 0, \ \ \ c.c.
			 \end{cases} $$
			 Sema $Z_1=X\cdot Y$ e $Z_2=\frac{X}{Y}$
			 \begin{enumerate}
			 	\item Determine a f.d.p conjunta de $Z_1$ e $Z_2$
			 	\item Determine a densidade de $X\cdot Y=Z_1$
			 \end{enumerate}
		 \underline{Sol}:
		 $$ 
		 g:\begin{cases}
		 z_1=x\cdot y\\
		 z_2=\frac{x}{y}
		 \end{cases}
		 \Rightarrow
		 det\bigg(J(x,y)\bigg)=det\begin{pmatrix}
		 y & x\\
		 \frac{1}{y} & -\frac{x}{2y}
		 \end{pmatrix}=-\frac{2x}{y}\ne 0
		 $$
		 
		 	 $$ 
		 g^{-1}=h:\begin{cases}
		 x=(z_1\cdot z_2)^{\frac{1}{2}}\\
		 y=\frac{z_1}{z_2}^\frac{1}{2}
		 \end{cases}
		 \Rightarrow
		 det\bigg(J(x,y)\bigg)=det\begin{pmatrix}
		 \frac{1}{2}(z_1\cdot z_2)^{-\frac{1}{2}}z_2 &  \frac{1}{2}(z_1\cdot z_2)^{-\frac{1}{2}}z_1\\
		 \frac{1}{2}\bigg(\frac{z_1}{z_2}\bigg)^{-\frac{1}{2}}\frac{1}{z_1} & \frac{1}{2}\bigg(\frac{z_1}{z_2}\bigg)^{-\frac{1}{2}}\frac{-z_1}{z_2} 
		 \end{pmatrix}
		 $$
		 $$=-\frac{1}{4z_2}-\frac{1}{4z_2}=\frac{1}{2z_2} $$
		 $$\Rightarrow det\bigg(
		 J(z_1,z_2)=-\frac{1}{2z_2}
		 \bigg) $$
		 Daí
		 
		 \begin{enumerate}[label=\alph*)]
		 	\item 
		 	$$f_{Z_1,Z_2}(z_1,z_2)= \begin{cases}
		 	\frac{1}{2z_1^2\cdot z_2}, \ \ \ \ \frac{1}{z_1}\le z_2\le z_1, \ \ z_1 \ge 1\\
		 	0,c.c
		 	\end{cases} $$
		 	$f_{Z_1,Z_2}(z_1,z_2)=f_{X,Y}\bigg(
		 	(z_1\cdot z_2)^{\frac{1}{2}},\bigg(\frac{z_1}{z_2}\bigg)^{\frac{1}{2}}
		 	\bigg)\cdot \bigg|
		 	\frac{-1}{2z_2}
		 	\bigg|$,
		 	\\
		 	\\
	
	$z_1\ge 1$, pois $x,y,\ge1$, $z_2=\frac{x}{y}$
	\item 
	$$f_{Z_1}(z_1)= \int\limits_{-\infty}^{\infty}f_{X,Y}\bigg(
	(z_1\cdot z_2)^{\frac{1}{2}},\bigg(\frac{z_1}{z_2}\bigg)^{\frac{1}{2}}
	\bigg)\cdot 
	\frac{1}{2z_2}dz_2
$$
$$
=\int\limits_{\frac{1}{z_1}}^{z_1} \bigg(
	\frac{z_2}{z_1z_2z_1}
	\bigg)
	\bigg(
	\frac{1}{2z_2}
	\bigg)
	dz_2=\frac{1}{2z_1^2}\int\limits_{\frac{1}{z_1}}^{z_1}\frac{1}{z_2}dz_2
$$
$$\Rightarrow f_{X\cdot Y}(z_1)= \frac{1}{2z_1^2}\bigg[ln(z_1)-ln(\frac{1}{z_1})\bigg] $$
$$\Rightarrow f_{X\cdot Y}(z_1)= \frac{1}{z_1^2}ln(z_1), \ \ \ z_1\ge 1 $$
		 \end{enumerate}
		 \item $\bm X$ abs cont. e g qualquer (não inversível)
		 $$ \bm Z=g(\bm X) \rightarrow f_{\bm Z} $$
		 
	\end{enumerate}



\newpage 

\section*{08/05}
\underline{Esperança}\\
\\
* Revisão (Integral de Riemann-Stiles)\\
\\
Seja $F:\mathbb R \longrightarrow \mathbb R$ f.d.a e $\phi: [a,b]\longrightarrow \mathbb R$ uma função borel-mensurável. A integral de R-S de $\varphi$ com respeito a F é definida por:

$$\int\limits_a^b \varphi(x)dF(x)
:= \lim\limits_{||\tau||} \sum\limits_i \varphi(y_i)[F(x_i)-F(x_i-1)]
 $$
 Sendo a partição de [a,b]
 
 $$=[x_1,x_2]\cup\cdots\cup[x_{n-1},x_n] $$
 
 $$\tau = \underset{1\le i\le n}{max} \{x_i-x_{i-1}\} \ \ e \ \ y_i \in [x_{i-1},x_i], \ i=1,\ldots,n$$
 Note que:
 $$F(x)=x $$
 Então
 $I=\int\limits_a^b \varphi(x)dx =\lim\limits_{||\tau||} \sum\limits_i \varphi(y_i)\Delta_{x_i}$ \bigg(
 Integral de Riemann
 \bigg)\\
\newpage 
 \underline{Integral de Lebesgue-Stieltjes}\\
 \\
 Seja:\\
 \\
 
$I = \int\limits_a^b \varphi(x)dF(x)$ Integral de R-S\\
\\
Resultado (Theorema de Caratheodory) \\
\\
Dada uma f.d. F, existe uma medida de probabilidade
$p_F: \mathscr A \longrightarrow [0,1] $ t.q.

$$F(x) = p_F\bigg((-\infty,x]\bigg) $$
em que $\mathscr A$  é $\sigma$-algebra do espaço amostral $\Omega$\\
\\
Utilizando o resultado anterior, tem-se que:

$$\int\limits_a^b \varphi(x)dF(x)= \int\limits_{\Omega} \varphi d\underbrace{p_F}_{medida \ de \ probabilidade} 
\ \ \ \bigg(Integral \ de \ Lebesgue-Stieltjes \bigg)
$$

\newpage 

\underline{Propriedade da integral de R-S}\\
\\

\begin{enumerate}[label=\arabic*)]
	\item  $$\int\limits_a^b c\varphi(x)dx = \lim\limits_{||\tau||}\sum\limits_{i=1}^n c\varphi(y_i)[F(x_i)-F(x_i-1)] $$
	$$ c\int\limits_a^b \varphi(x)dx = \lim\limits_{||\tau||}\sum\limits_{i=1}^n c\varphi(y_i)[F(x_i)-F(x_i-1)] $$
		$$=c\int\limits_a^b \varphi(x)dF(x)$$
		\item 
		$$ 
		\int\limits_a^b[\alpha \varphi_1(x)+ \beta \varphi_2(x)]dF(x)
		$$
				$$ 
		=\alpha \int\limits_a^b\varphi_1(x)dF(x)+ \beta  \int\limits_a^b\varphi_2(x)dF(x)
		$$
		
		\item $F$ fx
		\begin{itemize}
			\item F discreta com descontinuidade $\{x_1,x_2,\ldots,x_n\}$, Então:
			$$\int\limits_a^b \varphi(x)dF(x) = \sum\limits_{i=1}^{n} \varphi(x_i)[F(x_i)-F(x_i-1)] $$
			$$= \sum\limits_{i=1}^{n}\varphi(x_i)b_{x_i} $$
			Se X é uma variável aleatória tal que $X\sim F$, então
			$$\int\limits_a^b \varphi(x)dF(x) = \sum\limits_{i=1}^{n}\varphi(x_i)P(X=x_i)$$
			\item F absolutamente contínua com f.d.p f, então:
			$$\int\limits_a^b \varphi(x)dF(x)= \int\limits_a^b \varphi(x) f(x)dx $$
			
			\item $F=F_d +F_{ac} + F_S$
			$$ \int\limits_a^b \varphi(x)dF(x)= \sum\limits_{x_i \ desc. de F}\varphi(x_i)b_{x_i} + 
			\int\limits_a^b \varphi(x) f(x)dx 
			$$
		\end{itemize} 
	\newpage 
	\underline{Definição de Esperança}:
	Seja X uma variável aleatória definida sob o espaço de probabilidade $(\Omega,\mathscr{A},P)$ e F dua f.d.a.
	
	Seja $\varphi:\mathbb R\longrightarrow \mathbb R$ uma função Borel mensurável. A esperança de $\varphi(x)$ é definida por:
	
	$$E\bigg(\varphi(x)\bigg) =\int\limits_{\mathbb R} \varphi(x)dF(x) $$
	Sempre que ela \underline{existir}
	
	
	\underline{Obs:}
	$$E\bigg(\varphi(x)\bigg) =\int\limits_{-\infty}^{\infty} \varphi(x)dF(x) 
	=
	\underbrace{\int\limits_{-\infty}^{a} \varphi(x)dF(x) }_{(1)}
	+
	\underbrace{\int\limits_{a}^{\infty} \varphi(x)dF(x) }_{(2)}
	$$
	Existe, se:
	\begin{itemize}
		\item $(1)<\infty$ , $(2) < \infty \Rightarrow E\bigg(\varphi(x)\bigg)<\infty$
		
				\item $(1)=-\infty$ , $(2) < \infty \Rightarrow E\bigg(\varphi(x)\bigg)=-\infty$
				
						\item $(1)<\infty$, $(2) =+ \infty \Rightarrow E\bigg(\varphi(x)\bigg)=+\infty$
		
	\end{itemize}
	
\end{enumerate}
\newpage 
\underline{Exemplo}:\\
\\
X v.a.
$$E(X) = \int\limits_{\mathbb R }x dF(x)= \begin{cases}
\sum\limits_i x_iP(X=x_i), \ \ \ \ X \ dicreto\\
 \int\limits_{\mathbb R }x f(x)dx, \ \ \ \ X \ absolutamente \ continuo\\
\end{cases}
 $$

$$E(X^k) = \int\limits_{\mathbb R} x^k dF(x)= \begin{cases}
\sum\limits_i x_i^kP(X=x_i), \ \ \ \ X \ dicreto\\
\int\limits_{\mathbb R }x^k f(x)dx, \ \ \ \ X \ absolutamente \ continuo\\
\end{cases} $$

$$var(X) = E\bigg(X - E(X)\bigg)^2 $$
\underline{Exercícios}\\
\\
Calculdae a esperança e variância, se existir, de:
\begin{enumerate}[label = \arabic*)]
	\item $$X\sim Cauchy(1), \ \ f(x) = \frac{1}{(1+x)^2\pi}, \ x\in \mathbb R $$
	
	\item $$X\sim N(\mu,\sigma^2) $$
	
	\item $$X\sim Gamma(\alpha,\beta) $$
	\item $$X\sim Weibull(\alpha,\beta) $$
	\item $$X\sim t_{(n)} $$
	
	
\end{enumerate}

\newpage 

\underline{Exemplo}:

$$X\sim Exp(\lambda), \ \ \ \ Z=min\{X,Y\} $$
$$E(Z)=\int min\{X,\lambda\}f_X(x)dx 
= \int\limits_{0}^{\infty} min\{X,\lambda\}\lambda e^{-\lambda x}dx
$$

$$= \int\limits_{0}^{\lambda}x\lambda e^{-\lambda x}dx +
\int\limits_{\lambda}^{\infty} \lambda\cdot \lambda e^{-\lambda x}dx
$$

$$
= \bigg(-xe^{-\lambda x} \bigg)\bigg|_0^\lambda + \int\limits_0^\lambda e^{-\lambda x}dx +\lambda \bigg(
-e^{-\lambda x}
\bigg)\bigg|_\lambda^{+\infty}
$$

$$
= -\lambda e^{-\lambda^2} - \frac{1}{\lambda}\bigg(e^{-\lambda^2}-1\bigg) + \lambda e^{-\lambda^2} 
$$

$$
=\frac{1}{\lambda}\bigg(1-e^{-\lambda^2}\bigg)
$$
\underline{Proposição}:\\
\\
X é uma variável aleatória definida em $(\Omega,\mathscr{A},P)$, Então:

$$E\bigg(\varphi(x)\bigg) = \int\limits_0^\infty P(\varphi(X)>x)dx- \int\limits_{-\infty }^0 P(\varphi(X)\le x)dx $$

\underline{Obs:}

\begin{enumerate}[label=\arabic*)]
	\item  X discreta com valores inteiros:,
	
	$$E(X) = \sum\limits_{k=0}^{\infty} P(X>k) \Leftrightarrow E(X) =
	 \sum\limits_{k=0}^{\infty} kP(X=k) 
	$$
	
	
\end{enumerate}
\newpage
\section{13/05}
$$\int\limits_{0}^{\infty} x^{a-1} e^{-bx^c}dx = \frac{\Gamma\bigg(\frac{a}{c}\bigg)}{c(b)^{\frac{a}{c}}} $$
\underline{Exemplo}:
$$X\sim Exp(p) $$
$$E(X) = \int\limits_{0}^{\infty} P(X>x)dx  = \int\limits_x^{\infty} e^{-\beta x}dx$$
$$=\bigg(
-\frac{1}{\beta} e^{-\beta x}
\bigg)\bigg|_{0}^{\infty} $$
$$E(X^2) = \int\limits_0^\infty P(X^2>x)dx
=
\int\limits_0^\infty P(X>\sqrt x)dx
=
\int\limits_0^\infty e^{-\beta x^{\frac{1}{2}}}dx 
 = \frac{\Gamma(2)}{\beta^2\frac{1}{2}} 
 $$

X > 0 discreta com valores \{0,1,2,\ldots\}
$$E(X) = \sum\limits_{k=0}^{\infty} P(X>k) $$
\underline{Exemplo}:
$$X\sim Geo(p) $$

$$E(X) = \sum\limits_{k=0}^{\infty}P(X>k)= \sum\limits_{k=0}^{\infty} (1-p)^k = \frac{1}{1-(1-p)} = \frac{1}{p}$$
\underline{Exemplo}: (E1, pg. 189)\\
\\
X simétrica em torno de $\mu \Rightarrow E(X)=\mu$\\
\\

\underline{Prova}:\\
Hipótese: $P(X\ge x+\mu )= P(X\le x-\mu)$\\
Sugestão: use ($\alpha$), para $\mu=0$ e depois para $Y=X-\mu$ é imediato.

\newpage

\underline{Esperança de função de vetor aleatório}
\\

$\bm X = (X_1,\ldots,X_n):\Omega \longrightarrow\mathbb R^n $\\
\\
$g:\mathbb R^n \longrightarrow \mathbb R^n $  Borel mensurável 
\\
\\
$\Rightarrow Z=g(\bm X)$ é um vetor aleatório
e 

$$E(Z) = E\bigg(g(\bm X)\bigg)=
\int\limits_{\mathbb R^n}g(\bm x) dF(\bm x)
=
\begin{cases}
\sum\limits_{i_n}\cdots \sum\limits_{i1} g(\bm x_i)P(\bm X=\bm x_i)\\
\int\limits_{i_n}\cdots \int\limits_{i1} g(\bm x_i)f_{\bm X}(\bm x_i)dx_n\\

\end{cases}
$$

\underline{Exemplo}:\\
\\
\begin{enumerate}[label=\arabic*)]
	\item $X_1,\ldots,X_n$ variáveis aleatórias e $a_i \in \mathbb R^n, i=1,2,\ldots,n$, Então:
	
	$$E(\underbrace{a_1X_1+\cdots + a_nX_n}_{g(\bm X)}) =
	\int\limits_{\mathbb R^n} \cdots	\int\limits_{\mathbb R^n} \bigg(a_1x_1+\cdots a_nx_n\bigg) dF_{\bm X}(x_1,\ldots,x_n)
	$$
	
	
	$$
	= \int\limits_{\mathbb R^n} a_1x1	\int\limits_{\mathbb R^n} \cdots	\bigg(\int\limits_{\mathbb R^n} 
	f(x_1,\ldots,x_n) dx_n \bigg)dx_1
	+\cdots + 
	\int\limits_{\mathbb R^n} a_n x_n dx_n = a_1 E(X_1)+\cdots +a_n E(X_n)
	$$
	Isto é,
	$$E(a_1,X_1+\cdots a_nX_n) =  a_1 E(X_1)+\cdots +a_n E(X_n) $$
	\item $$E(X_1\cdots X_2) = \iint\limits_{\mathbb R^n} x_1 x_2 dF_{X_1,X_2}(x_1,x_2)$$
$$E\bigg(\frac{X_1}{X_2}\bigg) = \iint\limits_{\mathbb R^n} \frac{x_1}{x_2} dF_{X_1,X_2}(x_1,x_2)$$

\item X variável aleatória 
$$E\bigg[
X - E(X)
\bigg]^2 = var(X) $$

$$var(X) =  E\bigg[
X^2 -2XE(X) + \bigg(E(X)\bigg)^2
\bigg]$$
$$\therefore var(X) = E(X^2)-\bigg(E(X)\bigg)^2 $$

\item Medidas de dependência linear (Pearson)

$$\rho_{X_1,X_2} = \frac{Cov(X_1,X_2)}{\sqrt{var(X_1)}\sqrt{var(X_2)}} $$

$$cov(X_1,X_2)= 
E\bigg[
(X_1-E(X_1))\cdot(X_1-E(X_2))
\bigg]
 $$
 
 $$cov(X_1,X_2) = E(X_1X_2) - E(X_1)E(X_2) $$
 
 \item $X_1, X_2$ são independentes.
 $$\Rightarrow \rho_{X_1,X_2} = 0 $$  
 Pois $cov(X_1,X_2) = E(X1,X_2) -E(X_1)E(X_2) $
 
 \item $\rho_{X_1,X_2} = 0 \Rightarrow X_1, X_2$ são independentes?
 \item $X\sim B(n,p)$ 
 Defina:
  $$X = \sum\limits_{i=1}^{n} X_i  \begin{cases}
  X_i\sim Bernoulli\\
  E(X_i)=p, var(X_i)=p(1-p)\\
  X_i = \begin{cases}
  1,sucesso,P(X_i=1)=p\\
  0,fracasso, P(X_i=0)=1-p\\
  \end{cases}
  \end{cases}
  $$
  
  $$ 
  \Rightarrow E(X) = \sum\limits_{i=1}^{n} E(X_i)=\sum\limits_{i=1}^{n}p=np
  $$
  $$var\bigg(
  \sum\limits_{i=1}^{n} X_i
  \bigg)
  =np(1-p)
   $$
\end{enumerate}
\newpage 
\underline{Exercício}:
$$X\sim Hiper(N,r,n) $$
Prover que:

$$E(X)=\frac{nr}{N} $$
$$var(X) = \frac{nr}{N} \frac{N-r}{N}  \frac{N-n}{N-1}$$
\underline{Solução:}

$$
p_X(x)= \frac{\binom{r}{x}\binom{N-r}{n-x}}{\binom{N}{n}}
$$
\begin{itemize}
	\item $$
E(X) = \sum\limits_{x}x\frac{\binom{r}{x}\binom{N-r}{n-x}}{\binom{N}{n}}
$$
$$
x \binom{r}{x} =x \frac{r!}{(x)!(r-x)!} = \frac{r(r-1)!}{(x-1)!((r-1)-(x-1))!} = r \binom{r-1}{x-1}
$$

$$
 \binom{N-r}{n-x} = \frac{(N-r)!}{(N-r-n+x)!(n-x)!}= \frac{\bigg((N-1)-(r-1)\bigg)!}{\bigg(N-r-n+x\bigg)!\bigg((n-1)-(x-1)\bigg)!}$$

$$
=  \binom{(N-1)-(r-1)}{(n-1)-(x-1)}
$$

$$
\binom{N}{n} = \frac{N}{n}\binom{N-1}{n-1}
$$


$$ 
\therefore 
E(X) = \frac{rn}{N} \overbrace{\sum\limits_{x}\underbrace{\frac{\binom{r-1}{x-1}\binom{(N-1)-(r-1)}{(n-1)-(x-1)}}{\binom{N-1}{n-1}}}_{X\sim Hiper(N-1,r-1,n-1)}}^{1} \Rightarrow E(X) = \frac{rn}{N}
$$

\newpage 
\item
$$Var(X) = E(X^2) - E(X^2) $$
$$
E(X^2) = \sum\limits_{x}x^2\frac{\binom{r}{x}\binom{N-r}{n-x}}{\binom{N}{n}}
$$
Note que 
$$ E(X^2) = E(X(X-1)+X) $$
Utilizando uma argumentação similar a utilizada acima, tem-se:
$$
x(x-1)\binom{r}{x} =x(x-1) \frac{r!}{(x)!(r-x)!} = (x-1)\frac{r(r-1)!}{(x-1)!((r-1)-(x-1))!} $$
$$
=r(r-1) \frac{(r-2)!}{(x-2)!((r-2)-(x-2))!} = r(r-1)\binom{r-2}{x-2}
$$

$$
\binom{N-r}{n-x} = \frac{(N-r)!}{(N-r-n+x)!(n-x)!}= \frac{\bigg((N-2)-(r-2)\bigg)!}{\bigg(N-r-n+x\bigg)!\bigg((n-2)-(x-2)\bigg)!}$$

$$
\binom{N}{n} = \frac{N(N-1)}{n(n-1)}\binom{N-2}{n-2}
$$
$$ 
\therefore 
E(X(X-1)+X) = \frac{r(r-1)n(n-1)}{N(N-1)} \overbrace{\sum\limits_{x}\underbrace{\frac{\binom{r-2}{x-2}\binom{(N-2)-(r-2)}{(n-2)-(x-2)}}{\binom{N-2}{n-2}}}_{X\sim Hiper(N-2,r-2,n-2)}}^{1}
+ \frac{rn}{N}$$
$$
 \Rightarrow E(X^2) = \frac{r(r-1)n(n-1)}{N(N-1)} +  \frac{rn}{N}
$$

$$
Var(X) =  \frac{r(r-1)n(n-1)}{N(N-1)} +  \frac{rn}{N} - \bigg(\frac{rn}{N}\bigg)^2 = \frac{rn}{N}
\bigg(
\frac{(r-1)(n-1)}{(N-1)} + 1- \frac{rn}{N} 
\bigg)
$$
$$
 = \frac{rn}{N}
\bigg(
\frac{N(r-1)(n-1)}{N(N-1)} + \frac{N(N-1)}{N(N-1)}- \frac{rn(N-1)}{N(N-1)} 
\bigg)
$$
$$
 = \frac{rn}{N}
\bigg(
\frac{N(rn-n-r+1)}{N(N-1)} + \frac{N^2-N}{N(N-1)}- \frac{rnN-rn}{N(N-1)} 
\bigg)
$$
$$
= \frac{rn}{N}
\bigg(
\frac{\cancel{Nrn}-Nn-Nr\cancel{+N}}{N(N-1)} + \frac{N^2\cancel{-N}}{N(N-1)}- \frac{\cancel{Nrn}-rn}{N(N-1)} 
\bigg)
$$

$$
= \frac{rn}{N}
\bigg(
\frac{N^2-Nn-Nr+rn}{N(N-1)} 
\bigg)
=
 \frac{rn}{N}
\frac{(N-r)(N-n)}{N(N-1)} 
$$
\end{itemize}

\newpage

\section*{20/05}

\underline{Esperança}\\
\\
Seja $(\Omega,\mathscr{A},P)$ espaço de probabilidade e $A\in \mathscr A$

$\mathrm{I}_A : \underset{\omega \longrightarrow \mathrm I_A(\omega)}{\Omega \longrightarrow\{0,1\}} $ função Indicadora de A

$\mathrm I_A(\omega) = \begin{cases}
1, & \omega \in A\\
0 & \omega \not\in A
\end{cases}
\Leftrightarrow \begin{cases}
P(\mathrm I_A =1) = P(A)\\
P(\mathrm I_A=0) = P(A^c)
\end{cases}
$

$$\Rightarrow E(I_A)= \int\limits_{\mathbb R} x dF_{\mathrm I_A}(x)= \sum\limits_{x_i=0} x_i P(\mathrm{I}_A=x_i) $$
$$\Rightarrow E(\mathrm I_A)=P(A) $$

\underline{Esperança Condicional}

\begin{enumerate}[label=\roman*)]
	\item Esperança condicional dado um evento\\
	Seja $(\Omega,\mathscr{A},P)$ um espaço de probabilidade e $A,B, \in \mathscr A$, Então:
	
	$$P(A|B)=\frac{P(A,B)}{P(B)}, \ \ \ P(B)>0$$
	
	Por outro lado, seja $X:\Omega \longrightarrow \mathbb R$, uma variável aleatória, Então:\\
	\\
	
	A esperança condicional de X dado B é a função definida por:\\
	\\
	$$E(X|B) = \int\limits_{\mathbb R} xdF_{X}(x|B), \ \ \ \ \ \ (1) $$
	Sendo 
	$$F_X(x|B) = P(X\le x|B) = \frac{P(\{ X\le x\}\cap B)}{P(B)}$$
	A equação (1) pode ser reescrita como:
	
	$$E(X|B) = \int\limits_{\mathbb R} x  \frac{dP(\{ X\le x\}\cap B)}{P(B)} $$
	$$=\frac{1}{P(B)}\int\limits_{B} xdF_x(x) $$
	$$= \frac{1}{P(B)}E(X\cdot I_B) \ \ \ \ \ (2) $$
	$$\frac{1}{P(B)}\begin{cases}
	\sum\limits_{i:\{X=x_i\}\cap B}P(X=x_i), & X \ Discreto\\
	\int\limits_B xf(x)dx, & X \ Continuo
	\end{cases} $$
	\newpage 
	
	\underline{Exemplo}:
	
Seja $X\sim U[0,1]$ e $B_i = [\frac{i-1}{10},\frac{i}{10}]$ com   $\Omega = \bigcup\limits_{i=1}^{10} B_i$\\
\\
Determine $E(X|B_i)$.]\\
\\
\underline{Solução:}\\
\\
$\bigg(\underbrace{\Omega}_{[0,1]},\mathscr B([0,1]),P\bigg)$, Sendo P Uniforme

$$E(X|B_i) = \frac{1}{P(B_i)} \int\limits_{B_i} xf(x)dx = \frac{1}{\frac{1}{10}}\int\limits_{\frac{i-1}{10}}^{\frac{i}{10}}x\cdot 1 dx $$
$$E(X|B) = \frac{2i-1}{20}, \ \ \ 1=1,\ldots,10 $$

\underline{Exemplo 2}:

$X\sim U[-1,1]$. Calcule $E(X|X>0)$\\
\\
\underline{Solução:}
$$E(X|X>0) = \frac{1}{P(X>0)} \int\limits_{\{X>0\}} xf(x)dx = $$

$$= \frac{1}{\frac{1}{2}} \int\limits_0^1 x \frac{1}{2} dx =\frac{1}{2}$$
Pois

$$f_X(x) = \begin{cases}
\frac{1}{2}, & x \in [-1,1]\\
0, & c.c
\end{cases},
P(X>0) = \int\limits_0^1 \frac{1}{2}dx = \frac{1}{2}
 $$
 
 \item Esperança condicional dada uma variável aleatória discreta .\\
 \\
 Sejam $X:\Omega \longrightarrow \mathbb R $ e $Y:\Omega \longrightarrow \{y_1,y_2,\ldots\} $.\\
\\
Como

$$B_i = Y^{-1}(\{y_i\}) = \{Y=y_i\} \in \mathscr A, i=1,\ldots $$
Tal que $\bigcup_{i\ge 1}B_i = \bigcup\limits_{i\ge 1} \{Y=y_i\}=\Omega$. Então a função $E(X|Y)$ chamada de esperança condicional de Y, é tal que:

$$E(X|Y)(\omega) = E(X|Y=y_i)  = \frac{E\bigg( X\cdot I_{\{Y=y_i\}}\bigg)}{P(Y=y_i)}$$
Note que se X for discreta, $X:\Omega \longrightarrow \{x_1,x_2,\ldots \}$

$$E(X|Y)(\omega) = E(X|Y=y_i) = \frac{1}{P(Y=y_i)}\sum\limits_j x_j P(X=x_i,Y=y_i) $$
$$= \sum\limits_j x_j \frac{P(X=x_j,Y=y_i)}{P(Y=y_i)} $$

\underline{Questão:}

X e Y são absolutamente contínuas, $E(X|Y)=?$

$$E(X|Y)(\omega) = E(X|Y=y)= \int\limits_{\mathbb R} x \frac{f_{X,Y}(x,y)}{f_Y(y)}dx $$

\item $\sigma$-algebra gerada por uma variável aleatória

\begin{itemize}
	\item Considere\\
	\\
	$Y:\Omega \longrightarrow \{y_1,y_2,\ldots\}$ variável aleatória discreta, 
	Então
	
	$$B_i = y^{-1}(\{y_i\})= \{Y=y_i\} \in \mathscr A $$
	 e $\cup B_i=\Omega$. Uma $\sigma$-algebra que contém a coleção $\mathscr C =\{B_1,B_2,\ldots\}$\\
	É chamada de $\sigma$-algebra gerada por Y, denotada por $\sigma(Y)$.
	
	Temos que $\sigma(Y)$ é a coleção de uniões e intersecções finitas de eventos de $\mathscr C$ e seus complementares, Isto é:
	
	$$\sigma(Y) = \bigg\{
		\emptyset = \bigcup\limits_{i \in I=\emptyset}\{Y=y_i\}, \ldots, \mathscr C, \ldots, \Omega= \bigcup\limits_{i \in I=\mathbb N}\{Y=y_i\} \bigg\}
$$

I = conjunto de índices. Além disso, $\forall a,b,\in \mathbb R$,

$$\overbrace{Y^{-1} \bigg(
(a,b]
\bigg)}^{Info\ de \ Y} = \bigcup\limits_{i:a<y_i\le b} \{Y=y_i\} \in \sigma(y) $$

\item 
$Y:\Omega \longrightarrow \mathbb R$ absolutamente contínua.

$\sigma(Y)$ contém a coleção de Borelianos do tipo,
$$\forall a,b,\in \mathbb R, Y^{-1}\bigg((a,b]\bigg)\in \sigma(Y) $$
\item  Y é uma variável aleatória qualquer definida em $(\Omega,\mathscr{A},P)$

$$\Rightarrow \sigma(Y) \subset \mathscr A $$

A informação de Y está contida em $\mathscr A$


\end{itemize}

\item \underline{Esperança Condicional }

Seja X uma variável aleatória definida em $(\Omega,\mathscr{A},P)$. Uma variável aleatória denotada por $ W = E(X| \mathscr A_1 )\mathscr A_1\subset \mathscr A,$, é a esperança condicional de X dado $\mathscr A_1$, se :
\begin{enumerate}
	\item $\sigma(W)\subset \mathscr A_1$
	\item $E(X\cdot I_A) = E(W I_A), \forall A \in \mathscr A_1$ quase certamente (Exceto, A t.q. P(A)=0)
\end{enumerate}
* Se $E|X| < \infty$, tem-se que W existe e é única.
\underline{Exemplos}:

\begin{enumerate}[label = \arabic*)]
	\item  X variável aleatória e $Y:\Omega \longrightarrow \{y_1,y_2,\ldots\}$
	
	Temos que
	\begin{enumerate}[label=\roman*)]
		\item  $W=E(X|Y)$ e $W(\omega) = E(X|Y=y_i)=g(y_i)$
		$$\Rightarrow W=g(Y) \Rightarrow \sigma(W)\subset \sigma(Y) $$
		$$W=E(X|Y)=E(X|\sigma(Y)) $$
		
		\item Considere $A=\bigcup\limits_{i\in I}$, I = índices. Temos que:
		
		$$E(X\cdot I_A) = E\bigg(X\cdot \sum\limits_{i \in I}I_{\{Y=y_i\}}\bigg) $$
		$$=\sum\limits_{i \in I} E(X\cdot I_{\{Y=y_i\}}) $$
		Por outro lado, 
		$$E(\underbrace{W\cdot I_A}_{g(Y) \ discreta}) $$
		$$E(W\cdot I_A)=\sum\limits_i g(y_i)\cdot P(Y=y_i)  $$
		$$\sum\limits_i E(X|Y=y_i) P(Y=y_i) $$
		$$\sum\limits_i \frac{PE(X\cdot I_{\{Y=y_i\}})}{\cancel{P(Y=y_i)}}\cancel{P(Y=y_i)} $$
		$$=\sum\limits_i E(X\cdot I_{\{Y=y_i\}}) $$
		$$E(XI_A)=E\bigg[E(X|\mathscr A_1)I_A\bigg] $$
	\end{enumerate}
\end{enumerate}
\end{enumerate}

\newpage 

\section*{22/05}
E1 : $$Y:\Omega \longrightarrow\{y_1,y_2,\ldots\} $$
$$E(X|\sigma(Y)) = E(X|Y) = W $$
Note que:
$$E(X|Y)(\omega) = E(X|Y=y_i)\begin{cases}
\sum\limits_j x_j \frac{P(X=x_i,Y=y_i)}{P(Y=y_i)}, & X \ Discreta\\
\int\limits_{\{Y=y_i\}} x_i \frac{f_X(x)dx}{P(Y=y_i)}, & X \ Abs. \ Cont
\end{cases}
 $$
 
 E2:
 
 $\Omega$, $A,B\in \Omega $, $X:\Omega \longrightarrow \mathbb R$ e $\mathscr A_B=\{ \emptyset,B,B^c,\Omega \}=\sigma(B)$
\begin{itemize}
	\item  $W = E(X|\sigma(B))$ é tal que:
	
	$W(\omega) = E(X|B)  $ para $\omega in B$
	
	$W(\omega) = E(I_A|\sigma(B))=P(A|\sigma(B))$
	\item $X=I_A$
	$$W=E(I_A|\sigma(B)) = P(A|\sigma(B)) $$
\end{itemize}
Obs: X,Y variáveias aleatórias contínuas

$$E(X|Y)(\omega) = E(X|Y=y) $$
$$=\int\limits_\mathbb R x dF_{X|Y}(x|y) $$
$$=\int\limits_{\mathbb R} x \frac{f_{X,Y}(x,y)}{f_Y(y)}dx $$

\subsection{Propriedades}
$$\bigg(
X:\Omega \longrightarrow \mathbb R, (\Omega,\mathscr{A},P)
\bigg) $$

P1. $E(X) = E[E(X|\mathscr A_1)] $ , $\mathscr A_1 \subset \mathscr A$

Prova segue pela definição.

EX1:

$$E(X) = E[E(X|Y)] $$

$$=\begin{cases}
\sum\limits_{y_i} E(X|Y=y_i)P(Y=y_i), & Y \ discreta\\
\int\limits_{\mathbb R} E(X|Y=y)f_Y(y)dy, & Y \ ads. \ Cont.
\end{cases} $$
\newpage 
EX2

Considere X e Y com densidade conjunta

$$f_{X,Y}(x,y) =\frac{e^{-\frac{x}{y}} e^{-y}}{y}, x,y,>0 $$

Determine $E(X|Y)$

\underline{Solução}

$$E(X|Y=y) = \int\limits_0^\infty  x \frac{f_{X,Y}(x,y)}{f_Y(y)} dx
=\int\limits_0^\infty  x\frac{e^{-\frac{x}{y}} e^{-y}}{ye^{-y}} dx
$$

$$
= \int\limits_0^\infty \frac{x}{y} e^{-\frac{x}{y}}dx = y
$$

$$\therefore E(X|Y=y) = y $$
$$\leftrightarrow E(X|Y)=y\sim Exp(1) $$
P2. $X_1,X_2$ variáveis aleatórias definidas em $(\Omega,\mathscr{A},P)$, $\mathscr A_1\subset \mathscr A$ e $c_1,c_2 \in \mathbb R$, Então:

$$E(c_1X_1+c_2X_2|\mathscr A_1) = c_1E(X_1|\mathscr A_1) + c_2E(X_2|\mathscr A_1) $$

P3. Se X e Y são variáveis aleatórias independentes, então

$$E(X|\sigma(Y)) = E(X|Y) = E(X) $$
Em geral, se X e $\mathscr A_1$ são independentes,

$$E(X|\mathscr A_1)=E(X) $$

P4. Se $\sigma(X)\subset \mathscr A_1$, então

$$E(X|\mathscr A_1)=X $$

Neste caso, $\mathscr A_1$ fornece toda a estrutura de X (\underline{X se torna conhecida ou não aleatória})

\underline{Exemplo}:

Seja $X=g(y)$, $\sigma(x) = \sigma(g(y))\subset \sigma(Y)$

$$E\bigg(
g(y)|\sigma(y)
\bigg) = g(y) $$

p5. Se $\sigma(x)\subset \mathscr A_1$ e Z é uma variável aleatória qualquer, então

$$E(Z\cdot X|\mathscr A_1) = g(y)E(Z|\sigma(y)) $$

\underline{Exemplo}:
$$E(g(y)Z)|\sigma(Y)) = g(y)E(Z|\sigma(y))$$

\newpage 


\underline{Exemplo}:
$(X,Y)$,  $X\sim Exp\bigg(\frac{1}{2}\bigg)$ e $\forall x>0$ , $Y|X \sim U[0,x^2]$  

\begin{enumerate}[label=\alph*)]
	\item Distribuições de $Y|X^2=Z$
	$$ 
	F_{Y|x=x}=\begin{cases}
	0, & t<0\\
	\frac{t}{x^2} & 0\le t < x^2\\
	1 & t\ge x^2
	\end{cases}
	$$
	$$F_{Y|X^2} = P(Y\le y| X^2=x)=P(Y\le y|X=x^\frac{1}{2})$$
	$$F_{Y|X=x^\frac{1}{2}}= \begin{cases}
	0, & y<0\\
	\frac{y}{x}, & 0\le y < x\\
	1, & y>x
	\end{cases}
	 $$
	
	 $$Y|X^2=2 \sim  U[0,x] \Leftrightarrow Y|X \sim U[0,x]$$


\item $E(X) = \frac{1}{\frac{1}{2}}=2$
$$E(Y) = E[E(Y|X^2)] = E[\frac{X}{2}] \frac{1}{2}E(X)=1$$

$$E(XY)= E[E(XY|X^2)] \underbracket{=}_{P5} E(XE(Y|X^2))$$
$$=E[X\frac{X}{2}] = \frac{1}{2}E(X^2) $$
$$\frac{1}{2} \bigg[var(X) - E(X)^2 \bigg] $$
$$=\frac{1}{2}\bigg[4+4\bigg]=4 $$

$$\therefore Cov(XY)=4 $$

\end{enumerate}

\newpage 

\underline{Exemplo}:

$$E(X|\sigma(Y)) $$
$X_1,X_2,\ldots,X_n $  variáveis aleatórias independentes e $S_n = \sum\limits_{i=1}^{n}X_i$
$$E(S_{n+1} \mathscr A_n)  = E(S_n+X_{n+1}|\mathscr A_n) $$
$$\overbracket{=}^{P2} E(S_n|\mathscr A_n) + E(X_{n+1}|\mathscr A_n) $$ 
$$\overbracket{=}^{P3}S_n + E(X_{n+1}) $$

Se $E(X_{n+1})=0$
$$E(S_{n+1}|\mathscr A_n) = S_n $$
$\Leftrightarrow \{ S_n,\mathscr A\} $ é uma Martingale.
\end{document}